\begin{spacing}{0.980}
	\chapter{Conclusion \& Outlook}
	\label{ch:conclusion}
	
	Throughout this thesis, models of astrophysical sources are developed to estimate the relative neutrino contribution from
	decays of charmed hadrons as well as pions and kaons. In order to do so, analytical parametrizations for cross sections as well as
	spectral distributions are taken from the literature and integrated numerically to obtain the results.
	
	For the scenario of a young magnetar, a process according to the relevant steps described in \cite{Carpio_2020} is reproduced. Significant
	discrepancies are identified and determined to be due to the inclusion of an effective optical depth factor. By omitting this coefficient,
	the findings better match those presented in \cite{Carpio_2020} except for a potentially erroneous maximum proton energy, which would affect
	all subsequent results. Neutrinos from charm generally dominate at high energies or early times due to cooling factors suppressing
	longer-lived pions and kaons. In the magnetar case, charmed hadron contributions are not clearly distinguishable from other components,
	which contradicts
	the clear separation observed in \cite{Carpio_2020}. Natural extensions to this result are the inclusion of secondary muons and
	the calculation of a diffuse neutrino flux similar to \cite{Carpio_2020}. Potential improvements to the astrophysical model
	could be realized in terms of a more complex ejecta shell or by considering additional spindown mechanisms such as gravitational
	waves. The optical depth introducing significant inconsistencies represents the most glaring issue for which no obvious solution is
	provided, and should therefore be investigated further, preferably through direct comparison with the explicit implementation used in
	\cite{Carpio_2020}. This would also require replacing the parametrizations adopted for the present thesis with event
	generator results used in \cite{Carpio_2020} that are unfortunately inaccessible. Alternatively, the procedure
	outlined in Section \ref{sub:generators} enables the computation of cross sections independent from external references.
	Without these prerequisites, one cannot conclude whether the strong disagreement between our results and those in \cite{Carpio_2020}
	stems from the correct implementation of different methods or actual methodological errors. Until this has been resolved, the
	findings of this thesis should be regarded as being of purely qualitative nature.
	
	Transferring the approach to an AGN accretion disk setting produces results in line with prior expectations. At high energies,
	charm decays dominate the neutrino fluence and are cleanly separable from other contributions. This reflects the highly simplified
	model, which basically consists of a generic proton target with some astrophysical motivation behind it. Including synchrotron
	losses resulting from the presence of magnetic fields as well as the intense thermal radiation described in Section \ref{sub:accretion}
	as characteristic of AGN environments almost certainly affects the results obtained in a significant manner, potentially
	obscuring the accretion disk as a neutrino source altogether. Should this be the case, other source regions such as AGN jets
	might present more promising candidates for detecting neutrinos from charm decays \cite{Murase_2023}. As before, results must
	further be qualified with regard to the uncertain accuracy of the underlying parametrizations. Replacing these with event generator
	data would put the findings on a more solid foundation by calculating cross sections using one comparatively reliable method.
	\enlargethispage*{2\baselineskip}\newpage
\end{spacing}

Overall, the contents of this thesis offer intriguing insights into the rich field of astroparticle physics and encourage deeper
exploration of the broad range of relevant topics. The methods implemented are subject to further optimization, both in terms
of more accurate parametrizations as well as better execution efficiency, although the procedures already are reasonably flexible,
enabling future research by being easily adaptable to more sophisticated models. In summary, this thesis serves as an introductory
work into the field of multimessenger astrophysics and shows that the study of exotic decays via cosmic neutrinos is a worthwhile
endeavor that may well lead to new discoveries in the study of energetic astrophysical processes and sources.
