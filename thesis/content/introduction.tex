\chapter{Introduction}
\label{ch:introduction}

In their quest for a deeper understanding of astrophysical phenomena, researchers have moved to investigating
a combination of different signal types. This has given rise to the independent field of multimessenger astronomy
that studies electromagnetic radiation, cosmic rays, neutrinos and gravitational waves to infer which processes
drive their sources \cite{Meszaros_2019}. While such environments are likely opaque to photons, charged
particles may escape after being accelerated, creating a non-thermal \emph{Ultra-High-Energy Cosmic Ray}
({\kern+0.5pt}UHECR) population \cite{Becker_2008}. Experiments do indeed support this expectation by regular
measurements of energies from $10^{9}{\kern+1.5pt}\unit{\giga\electronvolt}$ up to
$10^{1{\kern-0.3pt}1}{\kern+1.5pt}\unit{\giga\electronvolt}$ in accordance with the
\emph{Greisen-Zatsepin-Kuzmin} (GZK{\kern+0.5pt}) cutoff \cite{pa}. The question then
becomes if it is possible to reconstruct any source region properties from these signals.

One basic constraint derived from observations of extreme energies is a relationship between the sizes and field
strengths of possible sources \cite{Hillas_1984}. From this follows that plausible candidates for UHECR origins
consist mostly of extragalactic regions like \emph{Active Galactic Nuclei} (AGN{\kern+0.5pt}) that are powered
by \emph{Super{\kern+0.4pt}-Massive Black Hole} (SMBH{\kern+0.5pt}) engines at their center, as well as more exotic
scenarios involving stellar remnants such as white dwarfs or neutron stars \cite{Tjus_2020, Gabici_2019, Drury_2012}.
Matching cosmic ray signals to their sources, however, proves to be an extremely difficult task, as large scale
turbulent magnetic fields isotropize the arrival directions of charged messenger particles.

This leaves neutrinos and gravitational waves as multimessenger candidates, both of which are the subjects of multiple
promising discoveries at high-{\kern+0.25pt}energy detection facilities \cite{ic_first_evidence, ic_more_evidence, ligo}.
Although gravitational signals provide valuable insights regarding the interaction and structure of massive objects,
they do not carry much information on particle acceleration. Neutrinos, on the other hand, are produced in hadronic
reactions with proton or photon fields close to the accelerating regions. Their weakly interacting nature allows them
to escape and propagate almost freely, making them reliable pointers to their sources \cite{Becker_2008}.

Due to this strong physical connection between highly energetic neutrinos and hadronic cosmic rays, they are thought to have
closely related origins. Recent combined analyses indicating simultaneous neutrino and gamma emissions from a
flaring blazar seem to be in agreement with this assumption for the case of an AGN scenario \cite{ic_blazar_flare, ic_blazar_signal}.
More compact sources such as strongly magnetized pulsars are also an area that is being actively researched \cite{Murase_2009}.
The standard approach to modeling these sites starts with injecting protons accelerated by some astrophysical process into
a photonic or baryonic target surrounding the potential source. In the resulting interactions, energetic pions and kaons are
produced, which then decay to neutrinos. This method is generally self-{\kern+0.25pt}consistent and capable of producing
plausible results, but neglects certain effects that might reveal more detailed information about the respective source
\cite{Carpio_2020}.

When working towards describing sources of neutrinos in the upper UHECR energy ranges, intermediate hadrons must exhibit
highly relativistic velocities. Accordingly, these particles experience extreme time dilation, leading to lifetimes
appearing several orders of magnitude longer than in their rest frames when viewed from the surrounding fields.
If sufficiently high matter densities are present, the typically investigated pions and kaons are therefore inelastically
scattered to lower energies before decaying to neutrinos. Due to their much shorter lifetimes, more exotic particles, many
of which contain at least one charm quark, then become significant contributors to the neutrino flux that would otherwise
remain suppressed \cite{Tjus_2023}.

The present thesis estimates the relative neutrino contribution from decays of such charmed hadrons. For this purpose, results
of the magnetar case presented in \cite{Carpio_2020} are reproduced before applying the same procedure to a simplified AGN
accretion disk model, assessing the feasibility of charm decay dominance in this latter scenario.

Following this introduction, Chapter \ref{ch:background} summarizes the core physical concepts relevant to this thesis,
including particle physics, multimessenger astronomy, and their connection to the treated astrophysical sources. Building
on this background, Chapter \ref{ch:methods} introduces semianalytical methods involving empirical parametrizations,
which are then employed for all further computations. Lastly, Chapter \ref{ch:results} presents the results, with Chapter
\ref{ch:conclusion} providing a concluding discussion as well as potential future prospects based on these findings.

On the topic of unit conventions, this thesis uses varying styles for different contexts. Physical equations are usually given
in Gaussian units, while masses might have natural units of energy when concerning particle physics or be written as multiples
of the solar mass in astrophysical settings. To avoid ambiguities, units will be clarified in the text whenever necessary.
