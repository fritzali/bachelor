\chapter{Introduction}
\label{ch:introduction}

In their quest for a deeper understanding of astrophysical phenomena, researchers have moved to investigating
a combination of different signal types. This has given rise to the independent field of multimessenger astronomy
that studies electromagnetic radiation, cosmic rays, neutrinos and gravitational waves to infer which processes
drive their sources \cite{Meszaros_2019}. While such environments are likely opaque to photons, charged
particles may escape after being accelerated, predicting a non-thermal \emph{Ultra-High-Energy Cosmic Ray}
({\kern+0.5pt}UHECR) population. Experiments indeed support this expectation by regular measurements of
energies from $1{\kern-0.3pt}0^{9}{\kern+0.5pt}\unit{\giga\electronvolt}$ up to
$1{\kern-0.3pt}0^{1{\kern-0.6pt}1}{\kern+0.5pt}\unit{\giga\electronvolt}$ in accordance with the
\emph{Greisen-Zatsepin-Kuzmin} (GZK{\kern+0.5pt}) cutoff \cite{pierre_auger_cosmic_rays}. The question then becomes
whether it is possible to reconstruct any source region properties from these signals.

One basic constraint derived from observations of extreme energies is a relationship between the sizes and field
strenghts of possible sources \cite{Hillas_1984}. From this follows that plausible candidates for UHECR origins
consist mostly of stellar remnants like white dwarfs or neutron stars as well as \emph{Active Galactic Nuclei}
(AGN{\kern+0.5pt}) that are powered by \emph{Super{\kern+0.4pt}-Massive Black Hole} (SMBH{\kern+0.5pt}) central engines
\cite{Tjus_2020, Gabici_2019, Drury_2012}. Matching of particular cosmic rays to specific sources, however,
proves to be extremely difficult as a consequence of large scale magnetic fields isotropizing 
