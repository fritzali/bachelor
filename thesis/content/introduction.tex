\chapter{Introduction}
\label{ch:introduction}

In their quest for a deeper understanding of astrophysical phenomena, researchers have moved to investigating
a combination of different signal types. This has given rise to the independent field of multimessenger astronomy
that studies electromagnetic radiation, cosmic rays, neutrinos and gravitational waves to infer which processes
drive cosmic sources \cite{Meszaros_2019}. While such environments are likely opaque to photons, charged particles
may escape after being accelerated, predicting a non-thermal \emph{Ultra-High-Energy Cosmic Ray} ({\kern+0.5pt}UHECR)
population. Experiments indeed support this expectation by measuring a relative abundance of energies up to
\qty{e12}{\giga\electronvolt} in accordance with the \emph{Greisen-Zatsepin-Kuzmin} (GZK{\kern+0.5pt}) cutoff
\cite{pierre_auger_cosmic_rays}.

The question now becomes whether it is possible to reconstruct any source region properties from these signals.
