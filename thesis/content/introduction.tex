\chapter{Introduction}
\label{ch:introduction}

In their quest for a deeper understanding of astrophysical phenomena, researchers have moved to investigating
a combination of different signal types. This has given rise to the independent field of multimessenger astronomy
that studies electromagnetic radiation, cosmic rays, neutrinos and gravitational waves to infer which processes
drive their sources \cite{Meszaros_2019}. While such environments are likely opaque to photons, charged
particles may escape after being accelerated, predicting a non-thermal \emph{Ultra-High-Energy Cosmic Ray}
({\kern+0.5pt}UHECR) population. Experiments indeed support this expectation by regular measurements of
energies from $10^{9}{\kern+0.5pt}\unit{\giga\electronvolt}$ up to
$10^{1{\kern-0.3pt}1}{\kern+0.5pt}\unit{\giga\electronvolt}$ in accordance
with the \emph{Greisen-Zatsepin-Kuzmin} (GZK{\kern+0.5pt}) cutoff \cite{pa}. The question then becomes
whether it is possible to reconstruct any source region properties from these signals.

One basic constraint derived from observations of extreme energies is a relationship between the sizes and field
strenghts of possible sources \cite{Hillas_1984}. From this follows that plausible candidates for UHECR origins
consist mostly of stellar remnants like white dwarfs or neutron stars as well as \emph{Active Galactic Nuclei}
(AGN{\kern+0.5pt}) that are powered by \emph{Super{\kern+0.4pt}-Massive Black Hole} (SMBH{\kern+0.5pt}) engines
at their center \cite{Tjus_2020, Gabici_2019, Drury_2012}. Matching of cosmic ray signals to specific sources,
however, proves to be an extremely difficult task because large scale magnetic fields isotropize the directions
of charged messenger particles.

This leaves neutrinos and gravitational waves as multimessenger candidates, both of which are the subject of multiple
promising discoveries at high-{\kern+0.25pt}energy detection facilities \cite{ic_first_evidence, ic_more_evidence, ligo}.
Although gravitational signals provide valuable insights regarding the interaction and structure of massive objects,
they do not carry information on particle acceleration. Neutrinos, on the other hand, are produced in hadronic
reactions with proton or photon fields close to the accelerating regions. Their weakly interacting nature allows them
to escape and propagate almost freely, making them reliable pointers to their sources \cite{Becker_2008}.

Due to their strong physical connection, the origins of highly energetic neutrinos are thought to be closely related
to those of hadronic cosmic rays. Recent combined analyses indicating simultaneous neutrino and gamma emission from a
flaring blazar seem to agree with the above assertion in case of an AGN scenario \cite{ic_blazar_flare, ic_blazar_signal}.

\cite{Murase_2009}
