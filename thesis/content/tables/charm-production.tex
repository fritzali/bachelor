\begin{table}[H]
	\centering
	\vspace{1.5ex}
	\caption[Parametrization of the charm quark differential cross section.]{Parametrization of the inclusive charm quark
			 production differential cross section. Coefficients are calculated from \cite{Goncalves_2007} to write $E_p$
			 in units of \unit{\giga\electronvolt} without needing redundant conversion steps. The exponent $m = \kern-0.5pt \num{1.2}$
			 is a constant at all energies. For the application at hand, energy ranges beyond the given validity intervals
			 are used as mentioned in the text.}
	\label{tab:charm-production}
	\sisetup{group-digits=integer, table-format=1.3}
	\begin{tabular}{l S[table-format=3.0] S[table-format=4.0] S S S S}
		\midrule\midrule
		{$E_p \mathbin{/} \unit{\giga\electronvolt}$} & {$a_1 \mathbin{/} \unit{\micro\barn}$} &
		{$a_2  \mathbin{/} \unit{\micro\barn}$} & {$b_1$} & {$b_2$} & {$n_1$} & {$n_2$} \\
		\midrule
		{$10^4 - 10^8$} & 826 & 8411 & 0.197 & 0.016 & 8.486 & 0.107 \\
		{$10^8 - 10^{1 \kern-0.3pt 1}$} & 403 & 2002 & 0.237 & 0.023 & 7.639 & 0.102 \\
		\midrule\midrule
	\end{tabular}
\end{table}
