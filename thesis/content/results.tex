\chapter{Results}
\label{ch:results}

Presented in this chapter are the results obtained by applying the previously described methods. Due to uncertain normalization
factors, all values are given with respect to the maximum charm component in terms of flux $\dot{\phi}_\nu$ or fluence $\phi_\nu$
as physical quantities. By its definition,
\begin{equation}
	\dot{\phi}_\nu \kern+0.25pt = \frac{d \kern+0.75pt \dot{N}_{\kern-0.5pt \nu} / \kern-1.0pt dE_\nu \kern+0.25pt}{4\pi d^2} \:,
	\label{eqn:flux}
\end{equation}
the flux counts neutrinos per energy, time and area. Equation \eqref{eqn:flux} derives from evenly spreading all spectral
intensity over a spherical surface with radius equal to the distance $d \kern+0.5pt$ from a single source. Integrating over
time yields the fluence,
\begin{equation}
	\phi_\nu \kern+0.25pt = \frac{dN_{\kern-0.5pt \nu} / \kern-1.0pt dE_\nu \kern+0.25pt}{4\pi d^2} \:,
	\label{eqn:fluence}
\end{equation}
which measures neutrino numbers per energy and area instead. Because $d \kern+0.5pt$ is a constant, forming ratios of
$\dot{\phi}_\nu$ or $\phi_\nu$ eliminates it. Accordingly, the following results are distance independent.

\begin{figure}[H]
	\centering
	\includegraphics{../plots/build/magnetar_charm_decay_comparison_with.pdf}
	\caption[Magnetar $\nu \kern+0.5pt$ flux from $c$ decay including optical depth.]
			{Comparison of individual charmed hadron contributions to the total charm neutrino flux at
			 $E_\nu = \kern-0.5pt \qty{e9}{\giga\electronvolt}$ from a newborn magnetar, including optical depth.}
	\label{fig:magnetar-charm-comparison-with}
\end{figure}


\begin{figure}[H]
	\centering
	\includegraphics{../plots/build/magnetar_neutrino_spectrum_with.pdf}
	\caption[Magnetar $\nu \kern+0.5pt$ flux compared to $c$ decay with optical depth.]
			{Effective neutrino light curves normalized to the maximum charm decay flux at
			 $E_\nu = \kern-0.5pt \qty{e9}{\giga\electronvolt}$ from a newborn magnetar, including optical depth.}
	\label{fig:magnetar-flux-with}
\end{figure}

\begin{figure}[H]
	\centering
	\includegraphics{../plots/build/magnetar_integrated_neutrino_spectrum_with.pdf}
	\caption[Magnetar $\nu \kern+0.5pt$ fluence compared to $c$ decay with optical depth.]
			{Expected neutrino fluence weighted by the charm contribution from a young magnetar for different time
			 intervals after formation, including optical depth.}
	\label{fig:magnetar-fluence-with}
\end{figure}


As mentioned earlier, neutrinos are not distinguished by flavor, but by the particle decay from which they
originate. Model parameters use the numerical values that section \ref{sec:computation} provides.

Shown in figure \ref{fig:magnetar-charm-comparison-with} is the temporal evolution of charmed hadron contributions to the
total charm component at $E_\nu = \qty{e9}{\giga\electronvolt}$ originating from a young magnetar.
Decays of $\smash{D^0}$ constitute most of the charmed neutrinos, with $\smash{D^+} \kern-0.5pt$ adding significant amounts
as well, especially at later times. Both $\smash{D^+_s}$ and $\smash{\Lambda^{\kern-0.5pt +}_{\kern+0.5pt c}}$ are roughly
the same, each contributing around \qty{10}{\percent} to the combined flux. This is in line with cross sections calculated
via \eqref{eqn:differential} or measured by \cite{lhc} as well as the respective branching fractions listed in table
\ref{tab:charm-hadrons} for effective three body decays to neutrinos.

Similarly, figure \ref{fig:magnetar-flux-with} presents light curves for pions and kaons next to the total charm contribution,
restricted to $\smash{E_\nu = \kern-0.5pt \qty{e9}{\giga\electronvolt}}$ neutrinos. As in \cite{Carpio_2020} from QCD calculations,
a factor of $1 \kern-0.5pt /3 - 3$ is adopted for the charmed hadron uncertainty and marked with a shaded blue band. Decays of
kaons generally contribute more than pions at this energy, with neutrinos from charm exceeding both by more than an order
of magnitude until about $\cramped{t = \kern-0.25pt \qty{e5}{\second}}$ after magnetar formation.

In order to evaluate the significance of different time periods, figure \ref{fig:magnetar-fluence-with} depicts integration
results over varying intervals for pion, kaon and charm fluence. Contrary to expectations, charmed hadron decays are dominant
at all energies by as much as one order of magnitude. Comparing figures \ref{fig:magnetar-flux-with} and
\ref{fig:magnetar-fluence-with} with the corresponding plots in \cite{Carpio_2020} reveals further inconsistencies. Notably,
light curves in figure \ref{fig:magnetar-flux-with} have relatively flat slopes at earlier times, and the charm neutrino fluence
in figure \ref{fig:magnetar-fluence-with} shows no decrease towards lower energies. Testing different potential error sources
indicates that the optical depth $\mathscr{O}$ is most likely responsible for these discrepancies. Inserting proportionalities
of ejecta density $\smash{n_\text{ej} \kern-0.5pt \propto t^{-3}}$ and radius $\smash{r_\text{ej} \kern-0.5pt \propto t}$ into
\eqref{eqn:optical} while assuming $\sigma_{p \kern-0.1pt p}$ to be constant leads to an approximate
$\smash{\mathscr{O} \propto t^{-2}}$ dependence and results in neutrino numbers being distorted significantly towards larger values
at earlier times. To verify this finding, calculations are repeated under omission of the optical depth.

\begin{figure}[H]
	\centering
	\includegraphics{../plots/build/magnetar_charm_decay_comparison_without.pdf}
	\caption[Magnetar $\nu \kern+0.5pt$ flux from $c$ decay excluding optical depth.]
			{Comparison of individual charmed hadron contributions to the total charm neutrino flux at
			 $E_\nu = \kern-0.5pt \qty{e9}{\giga\electronvolt}$ from a newborn magnetar, excluding optical depth.}
	\label{fig:magnetar-charm-comparison-without}
\end{figure}


\begin{figure}[H]
	\centering
	\includegraphics{../plots/build/magnetar_neutrino_spectrum_without.pdf}
	\caption[Magnetar $\nu \kern+0.5pt$ flux compared to $c$ decay without optical depth.]
			{Effective neutrino light curves normalized to the maximum charm decay flux at
			 $E_\nu = \kern-0.5pt \qty{e9}{\giga\electronvolt}$ from a newborn magnetar, excluding the optical depth
			 defined by \eqref{eqn:optical} as a modification. Properties of hadronic components in agreement with
			 Figure \ref{fig:magnetar-flux-with} are observed. Different shapes that are also seen in Figure
			 \ref{fig:magnetar-charm-comparison-without} agree with cooling and the omission of an optical depth 
			 as Section \ref{sub:cooling} describes, and lead to pion and charm peaks with larger magnitudes.
			 The same shaded uncertainty band as in Figure \ref{fig:magnetar-flux-with} is adopted for charm decays.}
	\label{fig:magnetar-flux-without}
\end{figure}

\vspace{1.5\baselineskip}
\begin{figure}[H]
	\centering
	\includegraphics{../plots/build/magnetar_integrated_neutrino_spectrum_without.pdf}
	\caption[Magnetar $\nu \kern+0.5pt$ fluence compared to $c$ decay without optical depth.]
			{Expected neutrino fluence normalized to the maximum charm contribution from a young magnetar for different time
			 intervals after formation, excluding the optical~depth defined by \eqref{eqn:optical} as a modification.
			 In agreement with prior expectations derived from Sections \ref{sub:spindown} and \ref{sub:charm} for energy
			 scaling and mean lifetimes, respectively, first pion, then kaon, and eventually charm dominate, sorted
			 by increasing energy. The same uncertainty band as in Figure \ref{fig:magnetar-flux-with}
			 for charm decays as well as the scaling by $E_\nu^2$ from Figure \ref{fig:magnetar-fluence-with} are adopted.
			 With charmed hadrons being significant beyond $E_\nu = \kern-0.5pt \qty{e9}{\giga\electronvolt}$ and their lower
			 uncertainty never exceeding the kaon curve, it is unclear if substantial charm signals are likely.}
	\label{fig:magnetar-fluence-without}
\end{figure}


As suspected, these results more closely match those in \cite{Carpio_2020}. Starting with the comparison of charmed hadrons, figure
\ref{fig:magnetar-charm-comparison-without} agrees with the previous statements regarding $\smash{D^0}$ and $\smash{D^+} \kern-0.5pt$
dominating over $\smash{D^+_s}$ and $\smash{\Lambda^{\kern-0.5pt +}_{\kern+0.5pt c}}$ decays. By excluding the optical depth,
one can reveal additional information via the identification of comparatively narrow peaks. Their relative positions are consistent with
the different mean lifetimes that table \ref{tab:charm-hadrons} lists, an observation which is not accessible from figure
\ref{fig:magnetar-charm-comparison-with} due to flux curves resembling broad plateaus rather than clearly defined extrema. Due to shorter
decay times resulting in vanishing cooling factors \eqref{eqn:cooling} at higher energies, curves are shifted to reproduce slight offsets
in charmed hadron contribution.

Again comparing \cite{Carpio_2020} to figures \ref{fig:magnetar-flux-without} and \ref{fig:magnetar-fluence-without} indicates improved
agreement, with distribution shapes that are roughly similar in both cases. Because of differences in the approaches, some deviation is
to be expected. Reference \cite{Carpio_2020} uses event generator results to model pion and kaon cross sections, as well as QCD calculations
for the charm component, while this thesis collects various parametrizations from the literature for semianalytical computations instead.
Especially the almost identical shapes of pion and kaon light curves in figure \ref{fig:magnetar-flux-without} are most likely a consequence
of the simplifying assumption that their spectral distributions are linearly proportional except for a cutoff determined from their rest
masses. Despite this, neutrino fluxes from pion or kaon decay individually fit their counterparts in \cite{Carpio_2020} fairly well when
the contribution from muons is ignored, and their respective maxima are within acceptable margins compared to the charmed hadron curves in
figure \ref{fig:magnetar-flux-without} and reference \cite{Carpio_2020}.

There are, however, certain disagreements that are unlikely to arise purely from variations in the described methodologies. To begin with,
reference \cite{Carpio_2020} provides $\cramped{E^M} = \kern-0.5pt \smash{\qty{1.3e13}{\giga\electronvolt}}$ as an initial value for the
energy of injected protons, which disagrees with inserting the same default parameters into \eqref{eqn:mono} to find
$\cramped{E^M} = \kern-0.5pt \smash{\qty{1.7e12}{\giga\electronvolt}}$ instead, after conversion from \unit{erg} to \unit{\giga\electronvolt}
units. If the larger and seemingly erroneous value is actually used, it would imply energies an order of magnitude higher than in the
calculations presented here. Predictions from this include earlier light curve cutoffs, as energies become insufficient after less time
has passed, and potentially smaller separation between charm, pion and kaon components, because at each point in time, only lower
energy hadrons are available for neutrino production, resulting in a depression of otherwise possible peak values. A comparison to
\cite{Carpio_2020} shows that figure \ref{fig:magnetar-flux-without} displays both of these features. Smaller injection energies
further imply lower charm decay contributions to the total neutrino fluence, which figure \ref{fig:magnetar-fluence-without} seems
to support as well when compared to \cite{Carpio_2020}. Adding to this concern is the large divergence from figures
\ref{fig:magnetar-flux-with} and \ref{fig:magnetar-fluence-with} that most likely stems from the optical depth. Setting this
factor to be constant supposes a fixed fraction of protons colliding to produce hadrons, independent from density changes
due to an expanding ejecta volume. Such an assumption is difficult to justify, with the potential exception
of extremely high densities. As reference \cite{Carpio_2020} explicitly includes an optical depth in the same way
this thesis does, it is unclear how the wide disparities between neutrino spectra come about, especially
considering that figures \ref{fig:magnetar-flux-without} and \ref{fig:magnetar-fluence-without} without $\mathscr{O}$ present
better matches than figures \ref{fig:magnetar-flux-with} and \ref{fig:magnetar-fluence-with} with $\mathscr{O}$ to those in \cite{Carpio_2020}.
This issue represents a major incongruence and warrants further scrutiny, which is discussed in more detail by the next chapter. Due to their
better agreement with prior expectations and to extract physical implications from the results, calculations without optical depth are treated
as valid in the following paragraph.

With the used model parameters, number densities of $n_\text{ej} = \smash{\qty{3.1e18}{\per\centi\meter\cubed}}$ are predicted after one
spindown time $t_\text{sd} = \smash{\qty{3.2e3}{\second}}$ has passed.



Accordingly, figure \ref{eqn:magnetar-flux-without} confirms that charm decays dominate pion and kaon contributions at earlier times. Including
secondary muons and pions would lead to larger fluxes, but is unlikely to exceed the charmed hadron peak due to its curve being
shifted along the temporal axis. With $t_\text{sd} = \smash{\qty{3.2e3}{\second}}$ as the spindown time, charm neutrino flux reaches its peak.



\begin{figure}[H]
	\centering
	\includegraphics{../plots/build/nucleus_neutrino_spectrum.pdf}
	\caption[]{}
	\label{fig:nucleus-fluence}
\end{figure}

\begin{figure}[H]
	\centering
	\includegraphics{../plots/build/nucleus_neutrino_spectrum.pdf}
	\caption[AGN accretion disk $\nu \kern+0.5pt$ fluence compared to $c$ decay.]
			{Expected neutrino fluence normalized to the maximum charmed hadron contribution from an AGN accretion disk.
			 Differences between the shapes of pion and kaon components compared to charm decays result from the chosen
			 view, as the flat increase observed for charmed hadrons occurs at lower energies in the case of pions and kaons
			 due to their much longer lifetimes. These decay times are listed in Sections \ref{sub:scattering} and \ref{sub:charm}
			 with Figure \ref{fig:nucleus-charm-comparison} providing the reasoning for the effects of cooling. Charmed hadrons dominate
			 the fluence from $E_\nu = \kern-0.5pt \qty{e9}{\giga\electronvolt}$ and above, lining up with prior expectations. This
			 threshold is sensitive to varying densities, with lower values producing a shift towards higher energies. A hard cutoff
			 is enforced by $E_p = \kern-0.5pt \qty{e12}{\giga\electronvolt}$ as the maximum proton energy. The same shaded uncertainty
			 band as in Figure \ref{fig:magnetar-flux-with} for charm decays as well as the scaling by $E_\nu^2$ from
			 Figure \ref{fig:magnetar-fluence-with} are adopted.}
	\label{fig:nucleus-fluence}
\end{figure}

