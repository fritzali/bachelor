\begin{otherlanguage}{ngerman}
	\chapter*{Zusammenfassung}

	Mit dem Aufkommen der Multimessenger{\kern+0.2pt}-Astronomie zeichnen sich Neutrinos zunehmend als vielversprechende
	Botenteilchen zur Untersuchung hochenergetischer astrophysikalischer Prozesse ab. Bei sehr hohen Energien können Zerfälle
	exotischer Hadronen, die mindestens ein Charm-{\kern+0.3pt}Quark enthalten, aufgrund ihrer kurzen Lebensdauer dominante
	Beiträge zur gesamten Neutrinofluenz liefern. Ziel dieser Arbeit ist die Untersuchung der relativen Signifikanz solcher
	Charmed{\kern+0.1pt}-Hadronen im Kontext verschiedener astrophysikalischer Quellen. Dazu wurden junge und stark
	magnetisierte Neutronensterne sowie die Akkretionsscheiben aktiver Galaxienkerne als Quellregionen modelliert.
	Mit Hilfe analytischer Parametrisierungen und numerischer Integration wurden die so ermittelten Protonenspektren
	dann in Hadronen und schließlich Neutrinos übersetzt. Durch diese semianalytische Rechnung konnte qualitativ gezeigt
	werden, dass es möglich ist Charm-Beiträge zum Neutrinofluss bei hohen Energien zu isolieren. Es wurden
	Probleme mit dem Vorgehen identifiziert, die auf unvollständige Grundannahmen hindeuten. Diese
	sollten weiter untersucht und mit robusteren Methoden korrigiert werden.

\end{otherlanguage}



{\let\clearpage\relax\chapter*{Abstract}\label{ch:abstract}}

With the advent of multimessenger astronomy, neutrinos are increasingly emerging as promising messenger particles for the
study of highly energetic astrophysical processes. At very high energies, decays of exotic hadrons containing at least one
charm quark can represent dominant contributions to the total neutrino fluence due to their short lifetimes. The aim of this
thesis is to investigate the relative significance of such charmed hadrons in the context of different astrophysical sources.
For this purpose, young and strongly magnetized neutron stars as well as the accretion disks of active galactic nuclei were
modeled as source regions. With the aid of analytical parameterizations and numerical integration, the proton spectra obtained
in this way were then translated into hadrons and ultimately neutrinos. This semianalytical calculation qualitatively
demonstrated that it is possible to isolate charm contributions to the neutrino flux at high energies. Problems
with the procedure were identified that point to incomplete basic assumptions. These should be
further investigated and corrected with more robust methods.
