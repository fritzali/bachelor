\chapter*{Appendix}
\label{ch:appendix}
\addcontentsline{toc}{chapter}{Appendix}
\chaptermark{Appendix}

\section{Reference Frames}
\label{sec:frames}

Depending on the application, energies in particle physics are either given as viewed from a suitable rest frame or
independent from the choice of coordinate system altogether. One widely adopted formulation uses the Mandelstam variables
\begin{align*}
	s &= (p_1 + p_2)^2 \\
	t &= (p_1 - p_3)^2 \\
	u &= (p_1 - p_4)^2
\end{align*}
to assign different channels to scattering processes via the squared momentum carried by the exchanged mediating particle.
Implied in this context is the Minkowski inner product, making the above quantities manifestly Lorentz invariant.

When working with parametrizations defined for use in different subdisciplines it often becomes necessary to convert from
center of mass energies $\sqrt{s}\,$ to the energy $E$ of a projectile in the target rest frame. With $E^2 = \bm{P}^2 c^2 + M^2 c^4$
as well as momenta $P = (E, \bm{P}\,)$ and $p = (m c^2 , 0)$ one finds
\begin{equation*}
	s = (P + p)^2 = (E + m c^2)^2 - \bm{P}^2 = 2E m c^2 + m^2 c^4 + M^2 c^4
\end{equation*}
for the invariant mass. This relation is typically approximated as $s = 2E m c^2$ at high energies.


\section{Cross Sections}
\label{sec:cross}

\subsection*{Scattering}

\autocite{Fagundes_2012}

\subsection*{Production}

\subsection*{Decay}

\section{Spectral Distributions}
\label{sec:spectral}

\section{Implementation}
\label{sec:implementation}

\section{Pulsar Spindown}
\label{sec:spindown}
