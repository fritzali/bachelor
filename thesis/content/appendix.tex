\addchap{Appendix}
\label{ch:appendix}



\section{Reference Frames}
\label{sec:frames}

Depending on the application, energies in particle physics are either given as viewed from a suitable rest frame or
independent from the choice of coordinate system altogether. One widely adopted formulation uses the Mandelstam variables
\begin{align*}
	s = (p_1 \kern-0.5pt + p_2)^2 &&
	t = (p_1 \kern-0.5pt - p_3)^2 &&
	u = (p_1 \kern-0.5pt - p_4)^2
\end{align*}
to assign different channels in scattering processes via the squared momentum carried by the exchanged mediating particle.
Implied in this context is the Minkowski inner product, making the above quantities manifestly Lorentz invariant.

When working with parametrizations defined for use in different subdisciplines it often becomes necessary to convert from
center of mass energies $\sqrt{s \kern+1.0pt} \kern+1.0pt$ to the energy $\kern-0.5pt E \kern+1.0pt$ of a projectile in the target rest
frame. With $E^2 = \bm{P}^2 c^2 + M^2 c^4$ as well as momenta $P = (E \kern+1.0pt , \bm{P} \kern+0.75pt c)$ and
$p = (m c^2 \kern-1.0pt , 0)$ one finds
\begin{equation*}
	s = (P + p)^2 = (E + m c^2)^2 - \bm{P}^2 c^2 = 2E m c^2 + m^2 c^4 + M^2 c^4
\end{equation*}
for the invariant mass. This relation is typically approximated as $s = 2E m c^2$ at high energies.



\section{Cross Sections}
\label{sec:cross}



\subsection*{Scattering}



\subsection*{Production}

\begin{equation*}
	x_F \kern+1.0pt \frac{\raisebox{-0.5ex}{$d\sigma$}}{\raisebox{0.5ex}{$dx_F$}} (x_F \kern+0.5pt, \kern-0.5pt E_p)
	= a \kern+0.5pt x_F^b \kern+1.0pt (1 - x_F^m)^n
\end{equation*}

\cite{Goncalves_2007}

\begin{align*}
	a = a_1 \kern-1.0pt - a_2 \ln\bigl(E_p\bigr) && b = b_1 \kern-1.0pt - b_2 \ln\bigl(E_p\bigr) &&
	n = n_1 \kern-1.0pt - n_2 \ln\bigl(E_p\bigr)
\end{align*}

$m = \num{1.2}$ 

\begin{table}[H]
	\centering
	\vspace{1.5ex}
	\caption[Parametrization of the $c$ quark differential cross section.]
			{Parametrization of the weighted charm quark
			 production differential cross section. Coefficients are calculated from \cite{Goncalves_2007} to write $E_p$
			 in units of \unit{\giga\electronvolt} without needing redundant conversion steps. The exponent $m = \kern-0.5pt \num{1.2}$
			 is a constant at all energies. For the application at hand, energy ranges beyond the given validity intervals
			 are used as mentioned in the text.}
	\label{tab:charm-production}
	\sisetup{group-digits=integer, table-format=1.3}
	\begin{tabular}{l S[table-format=3.0] S[table-format=4.0] S S S S}
		\midrule\midrule
		{$E_p \mathbin{/} \unit{\giga\electronvolt}$} & {$a_1 \mathbin{/} \unit{\micro\barn}$} &
		{$a_2  \mathbin{/} \unit{\micro\barn}$} & {$b_1$} & {$b_2$} & {$n_1$} & {$n_2$} \\
		\midrule
		{$10^4 - 10^8$} & 826 & 8411 & 0.197 & 0.016 & 8.486 & 0.107 \\
		{$10^8 - 10^{1 \kern-0.3pt 1}$} & 403 & 2002 & 0.237 & 0.023 & 7.639 & 0.102 \\
		\midrule\midrule
	\end{tabular}
\end{table}


As in \cite{Bhattacharya_2015} it is assumed that the cross section scales linearly with the nucleon number, yielding
\begin{equation*}
	\frac{\raisebox{-0.5ex}{$d\sigma$}}{\raisebox{0.5ex}{$dx_c$}} (x_c \kern+0.5pt, \kern-0.5pt E_p) = A^{-1}
	\frac{\raisebox{-0.5ex}{$d\sigma$}}{\raisebox{0.5ex}{$dx_F$}} (x_c \kern+0.5pt, \kern-0.5pt E_p)
\end{equation*}
for the inclusive $pp \rightarrow cX$ production.

\qty{75}{\percent} nitrogen and \qty{25}{\percent} oxygen $A = \num{14.5}$

\begin{equation*}
	\frac{\raisebox{-0.5ex}{$d\sigma$}}{\raisebox{0.5ex}{$dx_E$}} (x_E \kern+0.5pt, \kern-0.5pt E_p) = \int_{\kern+0.5pt x_E}^1 dz \, z^{-1}
	\frac{\raisebox{-0.5ex}{$d\sigma$}}{\raisebox{0.5ex}{$dx_c$}} (x_c \kern+0.5pt, \kern-0.5pt E_p) \kern+1.0pt D^{\kern+0.5pt h}_c (z)
\end{equation*}



\begin{equation}
	D^{\kern+0.5pt h}_{c}(z) = \frac{N_h z \kern+1.0pt (1 - z)^{\kern+0.5pt 2}}
	{\bigl((1 - z)^{\kern+0.5pt 2} + \epsilon_h z \bigl)^{\raisebox{-1.5ex}{$^2$}}}
	\label{eqn:fragmentation}
\end{equation}



\subsection*{Decay}

\begin{equation*}
	\tilde{F}_{\kern+1.0pt h \kern+1.0pt \rightarrow \kern+1.0pt \nu \kern+1.0pt} = D_h^{-1} \kern+1.0pt \Bigl( 6b_ha_h^2 - 4a_h^3
	- 12\lambda_h^3 a_h + 12\lambda_h^2 y - 6b_h y^2 + 4 y^3 + 12 \lambda_h^2 \ln \bigl((1 - y) / \lambda_h \bigr) \Bigr)
\end{equation*}

\begin{align*}
	a_h = 1 - \lambda_h && b_h = 1 - 2\lambda_h &&
	D_h = 1 - 8 \lambda_h - 12\lambda_h^2 \ln \bigl( \lambda_h \bigr) + 8 \lambda_h^3 - \lambda_h^4
\end{align*}

\begin{table}[H]
	\centering
	\vspace{2.0ex}
	\caption[Coefficients for $c$ hadron production, cooling and decay.]
			{Coefficients for charm hadron production,
			 cooling and decay to neutrinos. All parameters $\epsilon_h$ are taken from leading order QCD fits
			 via the FF as defined and described in \cite{Kniehl_2006} with normalizations $N_h$ given by \cite{Carpio_2020}
			 to rescale the integration of \eqref{eqn:fragmentation} over $[0 \kern+0.5pt , \kern-1.5pt 1]$ to approximately match
			 the fractions $f_h$ provided in \cite{Lisovyi_2016} from measurements. Effective masses $\sqrt{\tilde{s}_h}$ and branching
			 fractions $\mathscr{F}_h$ are determined by \cite{Bugaev_1998} and \cite{Bhattacharya_2016} from fitting decay rates.
			 Mean lifetimes $\tau_h$ and masses $m_h$ are adopted from \cite{pdg} in the particle listings. Mass type
			 quantities use natural units.}
	\label{tab:charm-hadrons}
	\sisetup{group-digits=integer, table-format=1.2}
	\begin{tabular}{c S[table-format=1.4] S[table-format=1.5] S[table-format=4.0] S[table-format=1.3] S S}
		\midrule\midrule
		{$h$} & {$N_h$} & {$\epsilon_h$} & {$\tau_h \mathbin{/} \unit{\femto\second}$} & {$\mathscr{F}_h$} &
		{$\sqrt{\tilde{s}_h} \mathbin{/} \unit{\giga\electronvolt}$} & {$m_h \mathbin{/} \unit{\giga\electronvolt}$} \\
		\midrule
		{$D^{0}$} & 0.577 & 0.101 & 410 & 0.067 & 0.67 & 1.86 \\
		{$D^{+}$} & 0.238 & 0.104 & 1033 & 0.176 & 0.63 & 1.87 \\
		{$D^{+}_{s}$} & 0.0327 & 0.0322 & 501 & 0.065 & 0.84 & 1.97 \\
		{$\Lambda^{\kern-0.5pt +}_{\kern+0.5pt c}$} & 0.0067 & 0.00418 & 203 & 0.045 & 1.27 & 2.29 \\
		\midrule\midrule
	\end{tabular}
\end{table}




\section{Spectral Distributions}
\label{sec:spectral}



\subsection*{Pions \& Kaons}



\subsection*{Charm}



\section{Pulsar Spindown}
\label{sec:spindown}



\section{Stochastic Acceleration}
\label{sec:stochastic}

$N = N_0 p^k$

$E = E_0 f^k$

$\ln(x^k) = k\ln(x)$

\begin{equation*}
	\frac{\ln (N / N_0)}{\ln (E / E_0)} = \frac{\ln(p)}{\ln(f)}
\end{equation*}

\begin{equation*}
	N = N_0 \left( \frac{E}{E_0} \right)^{\ln(p) / \ln(f)}
\end{equation*}

\begin{equation*}
	\frac{dN}{dE} = \frac{N_0}{E_0} \left( \frac{E}{E_0} \right)^\alpha
\end{equation*}

\begin{equation*}
	\alpha = \frac{\ln(p)}{\ln(f)} - 1
\end{equation*}

\cite{Longair_2011}

$\alpha = -2$

$dN / dE \propto E^{-2}$



\section{Implementation}
\label{sec:implementation}

In order to calculate neutrino spectra from hadronic distributions, several integrals have to be computed. Discretizing
this task allows the general case
\begin{align*}
	F(x, y) &= \int dz \: G(x, z) \, H(z, y) \\
	\intertext{to be rewritten as a Riemann sum. Assuming $G$ and $H$ are integrable over a given interval,}
	F_{ij} &= \sum\nolimits_k D_{kk} \, G_{ik} \, H_{kj}
\end{align*}
converges to the exact solution for sufficiently small steps. Transforming variables
\begin{align*}
	&&&& x \rightarrow x_i && y \rightarrow y_j && z \rightarrow z_k &&&&
\end{align*}
and defining $D_{kk} = z_{k+1} \kern-1.0pt - z_k$ leads to the above notation. It is easily shown how this expression in terms of
indices translates to the product of corresponding matrices
\begin{equation*}
	\bm{F} = \bm{G} \, \bm{D} \, \bm{H}
\end{equation*}
as an equivalent formulation. Here the output $\bm{F} \in \mathbb{R}^{m \times n}$ is obtained from the inputs
$\bm{G} \in \mathbb{R}^{m \times l}$ and $\bm{H} \in \mathbb{R}^{l \times n}$ as well as the square matrix
$\bm{D} \in \mathbb{R}^{l \times l}$ that encodes all step sizes on its diagonal. These results enable a quick and
efficient implementation of the required calculations as program code, where array arithmetic operations can greatly
increase execution speed.\footnote{$\,$In service of reproducability, all implementations can be viewed in
\href{https://github.com/fritzali/bachelor}{this} repository.}
