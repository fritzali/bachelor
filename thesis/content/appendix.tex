\chapter*{Appendix}
\label{ch:appendix}
\addcontentsline{toc}{chapter}{Appendix}
\chaptermark{Appendix}

\section{Reference Frames}
\label{sec:frames}

Depending on the application, energies in particle physics are either given as viewed from a suitable rest frame or
independent from the choice of coordinate system altogether. One widely adopted formulation uses the Mandelstam variables
\begin{align*}
	s = (p_1 + p_2)^2 &&
	t = (p_1 - p_3)^2 &&
	u = (p_1 - p_4)^2
\end{align*}
to assign different channels in scattering processes via the squared momentum carried by the exchanged mediating particle.
Implied in this context is the Minkowski inner product, making the above quantities manifestly Lorentz invariant.

When working with parametrizations defined for use in different subdisciplines it often becomes necessary to convert from
center of mass energies $\sqrt{s \kern+1.0pt} \kern+1.0pt$ to the energy $\kern-0.5pt E \kern+1.0pt$ of a projectile in the target rest
frame. With $E^2 = \bm{P}^2 c^2 + M^2 c^4$ as well as momenta $P = (E \kern+1.0pt , \bm{P} \kern+0.75pt c)$ and
$p = (m c^2 \kern-1.0pt , 0)$ one finds
\begin{equation*}
	s = (P + p)^2 = (E + m c^2)^2 - \bm{P}^2 c^2 = 2E m c^2 + m^2 c^4 + M^2 c^4
\end{equation*}
for the invariant mass. This relation is typically approximated as $s = 2E m c^2$ at high energies.


\section{Cross Sections}
\label{sec:cross}

\subsection*{Scattering}

\subsection*{Production}

\subsection*{Decay}

\begin{table}
	\centering
	\caption[This is a short test caption.]{This is a very long caption for the sample table.}
	\label{tab:test}
	\begin{tabular}{c c c c c}
		\toprule
		a & b & c & d & e \\
		\midrule
		1 & 10 & 100 & 1000 & 11 \\
		2 & 20 & 200 & 2000 & 22 \\
		3 & 30 & 300 & 3000 & 33 \\
		\bottomrule
	\end{tabular}
\end{table}

\section{Spectral Distributions}
\label{sec:spectral}

\section{Pulsar Spindown}
\label{sec:spindown}

\section{Implementation}
\label{sec:implementation}

In order to calculate neutrino spectra from hadronic distributions, several integrals have to be computed. Discretizing
this task allows the general case
\begin{align*}
	F(x, y) &= \int dz \: G(x, z) \, H(z, y) \\
	\intertext{to be rewritten as a Riemann sum. Assuming $G$ and $H$ are integrable over a given interval,}
	F_{ij} &= \sum\nolimits_k D_{kk} \, G_{ik} \, H_{kj}
\end{align*}
converges to the exact solution for sufficiently small steps. Transforming variables
\begin{align*}
	&&&& x \rightarrow x_i && y \rightarrow y_j && z \rightarrow z_k &&&&
\end{align*}
and defining $D_{kk} = z_{k+1} - z_k$ leads to the above notation. It is easily shown how this expression in terms of
indices translates to the product of corresponding matrices
\begin{equation*}
	\bm{F} = \bm{G} \, \bm{D} \, \bm{H}
\end{equation*}
as an equivalent formulation. Here the output $\bm{F} \in \mathbb{R}^{m \times n}$ is obtained from the inputs
$\bm{G} \in \mathbb{R}^{m \times l}$ and $\bm{H} \in \mathbb{R}^{l \times n}$ as well as the square matrix
$\bm{D} \in \mathbb{R}^{l \times l}$ that encodes all step sizes on its diagonal. These results enable a quick and
efficient implementation of the required calculations as program code, where array arithmetic operations can greatly
increase execution speed.\footnote{In service of transparency and reproducability, all implementations can be viewed
in \href{https://github.com/fritzali/bachelor}{this} repository.}
