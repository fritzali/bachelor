\chapter{Background}
\label{ch:background}

On their journey through space, messenger particles are subject to various influences, both in terms of type and scale,
before reaching our planet. Understanding this propagation requires a thorough comprehension of physical processes at all
scales of application, from production in astrophysical sources, through interactions with the vast radiation and matter
fields that fill the cosmos, to entering the solar system and ultimately the terrestrial atmosphere.

The following sections provide an incomplete overview of aspects relevant to the treatment of this complex topic, as well as references
for further study on the various related subjects.



\section{Particle Physics}
\label{sec:particle}

In our modern view, interactions and categories of elementary particles are most accurately described by a construct called the
\emph{Standard Model} (SM) of particle physics. Underlying this formalism is the mathematical framework known as
\emph{Quantum Field Theory}~(QF$\kern+0.5pt$T) that combines and generalizes properties of
\emph{Quantum Mechanics}~(QM) and \emph{Special Relativity}~(SR) to produce a consistent description of microscopic reality.
Excitations in the associated fields are referred to as quanta and manifest themselves as observable particles.

Since this work is mostly concerned with the statistical behaviour of large quantities instead of individual particle probabilistics,
a more detailed treatment is omitted with the exception of certain phenomena important to the justification of some later assumptions.



\subsection{Reference Frames}
\label{sub:frames}

Depending on the application, energies in particle physics are either given as viewed from a suitable rest frame or independent from
the choice of coordinates altogether. One particularly convenient formulation uses incoming and outgoing momenta
$p_1 , \kern-1.0pt p_2$ and $p_3 \kern+0.5pt , \kern-1.0pt p_4$ to define
\begin{align*}
	s = \bigl( p_1 \kern-0.5pt + p_2 \bigr)^2 &&
	t = \bigl( p_1 \kern-0.5pt - p_3 \bigr)^2 &&
	u = \bigl( p_1 \kern-0.5pt - p_4 \bigr)^2
\end{align*}
as the Mandelstam variables to assign different channels in scattering processes via the squared momentum carried by the exchanged mediating particle.
Implied in this context is the Minkowski inner product, making the above quantities manifestly Lorentz invariant.

When working with parametrizations defined for use in different subdisciplines it often becomes necessary to convert from
center of mass energies $\sqrt{s \kern+1.0pt} \kern+1.0pt$ to the energy $\kern-0.5pt E \kern+1.0pt$ of a projectile in the
target rest frame. With $E^2 = \bm{P}^2 c^2 + M^2 c^4$ as well as momenta $P = (E \kern+1.0pt , \bm{P} \kern+0.75pt c)$ and
$p = (m c^2 \kern-1.0pt , 0)$ one finds
\begin{equation*}
	s = (P + p)^2 = (E + m c^2)^2 - \bm{P}^2 c^2 = 2E m c^2 + \kern+1.0pt m^2 c^4 + M^2 c^4
\end{equation*}
for the invariant mass. This relation is typically approximated as $s = 2E m c^2$ at high energies.



\subsection{Fundamental Interactions}
\label{sub:interactions}



\subsection{Hadrons \& Leptons}
\label{sub:hadrons}

Confinement, Asymptotic Freedom, Residual Nuclear Force, Decays



\subsection{Cooling \& Decay}
\label{sub:cooling}



\section{Multimessenger Astronomy}
\label{sec:multimessenger}



\subsection{Cosmic Rays}
\label{sub:rays}



\subsection{Neutrinos}
\label{sub:neutrinos}



\subsection{Photons}
\label{sub:photons}



\subsection{Gravitational Waves}
\label{sub:gravitational}



\section{Astrophysical Sources}
\label{sec:sources}



\subsection{Magnetic Field Scales}
\label{sub:fields}



\subsection{High Energy Cutoff}
\label{sub:cutoff}



\subsection{Magnetars}
\label{sub:magnetars}



\subsection{Pulsar Spindown}
\label{sub:spindown}



\subsection{Active Galactic Nuclei}
\label{sub:nuclei}



\subsection{Accretion Disks}
\label{sub:accretion}



\subsection{Stochastic Acceleration}
\label{sub:acceleration}
