\chapter{Background}
\label{ch:background}

While traversing space, messenger particles are subjected to many different influences, both in terms of type and magnitude, before being
detected at earth. Understanding this propagation requires a thorough comprehension of physical processes at all scales of application,
from production in astrophysical sources, through interactions with the vast radiation and matter fields that fill the cosmos, to entering
the solar system and finally the terrestial atmosphere.

The following sections provide an incomplete overview of aspects relevant to the treatment of this complex topic, as well as references
for further study on the various related subjects.



\section{Particle Physics}
\label{sec:particle}

In the modern view, interactions and categories of elementary particles are most accurately described by a construct called the
\emph{Standard Model}~(\abbrev{SM}) of particle physics. Underlying this formalism is the mathematical framework known as
\emph{Quantum Field Theory}~(\Abbrev{QFT}{QF$\kern+0.5pt$T}) that combines and generalizes properties of
\emph{Quantum Mechanics}~(\abbrev{QM}) and \emph{Special Relativity}~(\abbrev{SR}) to produce an extremely precise description
of microscopic reality. Excitations in the associated fields are referred to as quanta and manifest as observable particles.

Since this work is mostly concerned with the statistical behaviour of large quantities instead of individual particle probabilistics,
a more detailed explanation is omitted except for certain phenomena important to the justification of some later assumptions.



\subsection*{Leptons}
\label{sub:leptons}

Neutrinos, Decays



\subsection*{Hadrons}
\label{sub:hadrons}

Confinement, Asymptotic Freedom, Residual Nuclear Force, Decays



\section{Multimessenger Astronomy}
\label{sec:multimessenger}



\subsection*{Gravitational Waves}
\label{sub:gravitational}



\subsection*{Cosmic Rays}
\label{sub:rays}



\subsection*{Photons}
\label{sub:photons}



\subsection*{Neutrinos}
\label{sub:neutrinos}



\section{Astrophysical Sources}
\label{sec:sources}



\subsection*{Magnetars}
\label{sub:magnetars}



\subsection*{Active Galactic Nuclei}
\label{sub:nuclei}
