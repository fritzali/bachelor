\let\backupskip\chapterheadstartvskip
\renewcommand*\chapterheadstartvskip{\vspace*{3\topskip}} 
\chapter{Background}
\label{ch:background}
\let\chapterheadstartvskip\backupskip

On their journey through space, messenger particles are subject to various influences, both in terms of type and scale,
before reaching our planet. Understanding this propagation requires a thorough description of different physical processes
at all scales of application, from production in astrophysical sources, through interactions with the radiation and matter
fields that fill the cosmos, to entering the solar system and ultimately the terrestrial atmosphere. The following sections
provide an overview of aspects relevant to the treatment of this complex topic, as well as references for further study on
the various related subjects.



\section{Particle Physics}
\label{sec:particle}

In our modern view, interactions and categories of elementary particles are most accurately described by a construct called the
\emph{Standard Model} (SM) of particle physics. Underlying this formalism is the mathematical framework known as
\emph{Quantum Field Theory}~(QF{\kern+0.5pt}T{\kern+0.5pt}) that combines and generalizes properties of
\emph{Quantum Mechanics}~(QM) and \emph{Special Relativity}~(SR) to produce a consistent description of microscopic reality.
Excitations in the associated fields are referred to as quanta and manifest themselves as observable particles \cite{Peskin_1995}.
Since this thesis is mostly concerned with the statistical behaviors of large quantities instead of individual particle probabilities,
a more detailed treatment is omitted, with the exception of certain phenomena important to the justification of some later assumptions.



\subsection{Fundamental Interactions}
\label{sub:interactions}

With the theory of \emph{Quantum Electrodynamics} (QED) representing a quantized description of classical electromagnetism, the
first successful QF{\kern+0.5pt}T formulation had been realized. It governs all interactions between electrical charges that involve
photons and predicts observations with extremely high precision. Each subsequent QF{\kern+0.5pt}T is built on a foundation of QED
achievements, such as \emph{Quantum Chromodynamics} (QCD) for strong force interactions or the unified framework of
\emph{Electroweak Theory} (EWT) that includes both weak force and electromagnetic phenomena.

According to the SM formalism, these interactions are mediated by bosonic elementary particles, namely massless photons
$\gamma \kern+0.75pt$ for electrodynamics or gluons $\kern-0.5pt g$ in case of the strong force sector, as well as massive
$Z$ and $W^\pm$ bosons that carry the weak force. The only remaining fundamental interaction not contained in the SM is gravity,
the effects of which are currently best described in terms of \emph{General Relativity} (GR{\kern+0.25pt}) as the continuous
macroscopic curvature of spacetime due to the presence of large masses. Because elementary particles exclusively inhabit
microscopic scales, their behavior is usually indistinguishable from that described by SR in flat space \cite{Peskin_1995}.


\subsection{Hadrons \& Leptons}
\label{sub:hadrons}

In addition to the previously listed gauge bosons, the SM further includes elementary fermions as the fundamental constituents of
matter. These sort into the categories of quarks and leptons, both of which are further grouped by generations of quark or lepton
pairs. Ordered according to their generation, there are flavors of up and down, charm and strange, as well as top and bottom for
quarks, which are denoted by their initial letters. For charged leptons, there exist electrons $e$ as well as the heavier muon
$\kern-0.5pt \mu$ and tau $\kern-0.5pt \tau \kern+0.5pt$ that each have an associated neutrino $\nu_l \kern+0.5pt$ with no electrical
charge. This picture is completed by the corresponding antiquarks and antileptons for all particles mentioned above \cite{Halzen_1984}.

Apart from an electrical charge, quarks are also carriers of so-called color charges that couple to gluons, which, in contrast
to the electrically neutral photons, carry colors themselves. Gluons are therefore capable of self-{\kern+0.25pt}coupling,
leading to higher binding energies with increasing distances. If removed far enough, this energy is eventually
released by producing a quark pair, ultimately leading to the creation of bound color-neutral states.

These resulting composite particles are called hadrons, with the entire process being referred to as hadronization. This naturally
leads to the concept of confinement, which states that quarks cannot exist as free particles, with the notable exception of very
high energies. In these regimes, strong force coupling becomes small and allows quarks to behave almost freely, defining the term
asymptotic freedom. A major consequence of this is that perturbative QCD calculations become possible, which would otherwise
require complex approaches such as discrete QCD on a lattice or some type of effective field theory instead \cite{qwg}.

The quark model enables classification of hadrons based on their constituent numbers. Baryons are comprised of three quarks, all
with different colors, while mesons contain two quarks with opposite color and anticolor charges. From the spin algebra, baryons
clearly obey fermionic statistics, while mesons act as bosons. Exemplary quark contents are given for baryons such as protons
$p \equiv uud \kern+0.5pt$ and neutrons $n \equiv udd \kern+0.5pt$ as well as pions
$\pi^- \kern-1.0pt \equiv \kern+0.75pt \overline{\kern-0.5pt u \kern+0.5pt} d \kern+0.5pt$ and kaons
$K^- \kern-1.0pt \equiv \kern+0.75pt \overline{\kern-0.5pt u \kern+0.5pt} s \kern+0.5pt$ in the case of mesons.
The charmed hadrons considered in this thesis are the same mesons
$D^0 \kern-0.75pt \equiv c \kern+1.25pt \overline{\kern-0.5pt u \kern+0.5pt} \kern+0.5pt$ as well as
$D^- \kern-1.0pt \equiv \kern+0.5pt \overline{\kern-0.25pt c \kern+0.75pt} d \kern+0.5pt$
and $D^-_s \kern-1.0pt \equiv \kern+0.5pt \overline{\kern-0.25pt c \kern+0.75pt} s \kern+0.5pt$
with the baryon $\Lambda^{\kern-0.5pt +}_{\kern+0.5pt c} \kern-0.5pt \equiv u \kern+0.5pt d \kern+0.5pt c \kern+0.5pt$
as in \cite{Carpio_2020}. Specific decay channels will be listed directly in the relevant sections
with \cite{pdg} providing a broad overview.

Contrary to hadrons, leptons are fundamental particles. Due to being colorless, they are not affected by the strong force, meaning
that all leptonic interactions obey EWT instead. Noting that photons only couple to electrical charges, this leaves neutrinos as the
only particles which interact exclusively via the weak force. Here, one distinctive feature of the weak interaction should be
mentioned as well, namely that of parity violation. This manifests by only coupling to left-handed components of particles and
right-handed components of antiparticles. Handedness in this context refers to chirality, which for massless particles is equivalent
to helicity or the sign of the spin projection onto the momentum vector. Because the SM assigns zero mass to neutrinos, this EWT
property suppresses certain decay modes and enforces correlated polarizations for the involved leptons. Not included in the SM is
the observation of neutrino oscillations, which are, however, not relevant for this thesis, since flavor differences are neglected
\cite{Peskin_1995}.



\begin{spacing}{0.985}
	\subsection{Reference Frames}
	\label{sub:frames}

	Depending on the application, energies in particle physics are either given as viewed from a suitable rest frame or independent from
	the choice of coordinates altogether. One particularly convenient formulation uses incoming and outgoing momenta
	$p_1 , \kern-1.0pt p_2$ and $p_3 \kern+0.5pt , \kern-1.0pt p_4$ to define
	\begin{align*}
		s = \bigl( p_1 \kern-0.5pt + p_2 \bigr)^2 \: , &&
		t = \bigl( p_1 \kern-0.5pt - p_3 \bigr)^2 \: , &&
		u = \bigl( p_1 \kern-0.5pt - p_4 \bigr)^2
	\end{align*}
	as the Mandelstam variables $s, t, u,$ which assign different channels to scattering processes based on the squared momentum carried
	by an exchanged mediating particle. Implied in this context is a Minkowski inner product, making the above quantities manifestly Lorentz
	invariant. When working with parametrizations defined for use in different subdisciplines, it regularly becomes necessary to convert
	from center of mass energies $\sqrt{s \kern+1.0pt} \kern+1.0pt$ given by the square root of the Mandelstam $s$ to the energy
	$\kern-0.5pt E \kern+1.0pt$ of a projectile as viewed in the system of a stationary target, where vectors
	$P = (E \kern+1.0pt , \bm{P} \kern+0.75pt c)$ and $p = (m c^2 \kern-1.0pt , 0)$ represent the respective particle momenta.
	With a classical projectile momentum $\bm{P}$ and $c$ denoting the speed of light, rest masses $M$ and $m$ for projectile
	and target lead to $E^2 \kern-0.5pt = \bm{P}^2 c^2 + M^2 c^4$ and
	\begin{equation*}
		s = (P + p)^2 = (E + m c^2)^2 - \bm{P}^2 c^2 = 2E m c^2 + \kern+1.0pt m^2 c^4 + M^2 c^4 \: ,
	\end{equation*}
	which is also reffered to as the invariant mass. The trailing terms in this equation are negligible at high energies, allowing
	for an approximation $s = 2E m c^2$ in the target rest frame.



	\subsection{Particle Collisions}
	\label{sub:collisions}

	Another important aspect of particle physics is the description of collision processes. For this purpose, the concept of cross
	sections $\sigma \kern+0.5pt$ should be understood. In classical mechanics, these quantities are solely related to geometric
	properties of the objects involved, assuming they only interact upon impact. If longer range forces are included, the
	cross section represents a larger effective area that measures how the trajectory of an incoming projectile is influenced by
	the target. Additionally, differential cross sections with respect to some independent variable such as solid angle
	$d\sigma \kern-0.3pt / \kern-0.6pt d\Omega$ or kinetic energy $d\sigma \kern-0.3pt / \kern-0.6pt dE$ can be defined.
	Measurements of this quantity often reveal more information about the inner structure of targets. To distinguish these cases,
	one might refer to $\sigma$ as the integrated cross section. Since QM is at the core of particle physics, cross sections in this
	context are instead interpreted as probability measures of fundamentally stochastic collision processes, such as the production
	of specific particles. It can also be useful to separate cross sections into elastic and inelastic components, especially for
	collisions in which particles lose energy, as is the case for cooling processes via scattering.



	\subsection{Cooling \& Decay}
	\label{sub:cooling}

	Consider an infinitesimally thin slice of a target medium with particle number density $n$ and volume $V \kern-0.1pt = S \kern+1.5pt dx$
	where $S \kern+0.5pt$ and $dx$ measure surface area and thickness, respectively. Accordingly, there exist $\tilde{N} \kern-1.2pt = nV$
	targets that each have effective interaction cross sections $\tilde{\sigma}$ with a total coverage of $\tilde{S} = \tilde{\sigma} \tilde{N}$
	as viewed by an incident projectile.
	\enlargethispage{\baselineskip}\newpage
\end{spacing}

\begin{spacing}{0.980}
	Probabilities for dissipating the beam constituent energy are then given by the ratio $\mathscr{P} = \tilde{S} / S$ of both areas or
	explicitly $\mathscr{P} = \kern-0.5pt \tilde{\sigma} \kern+0.5pt n \kern+1.5pt dx$ in case of $dx$ as the covered distance.
	Expressing $\tilde{\sigma} = \kappa\sigma$ in terms of the inelastic scattering cross section $\sigma$ and a dimensionless factor
	$\kappa$ called inelasticity, one can identify a length scale $\lambda = (\kappa\sigma n)^{-1}$ as the mean free path between
	collisions. Multiplying with $\kappa$ includes the ratio of remaining to initial energy, which is taken to be constant. From this
	follows a reduction in beam particles $dN \kern+0.1pt = - N \lambda^{-1} dx$ proportional to $N \kern+1.0pt$ as the total projectile
	count and $\mathscr{P} = \kern-0.6pt \lambda^{-1} dx$ for the reformulated probability which represents an ordinary differential equation
	of first order. The solution is found to follow an exponential law
	\begin{equation*}
		N(x) \kern+0.4pt = N_0 \exp \Bigl( -\frac{\raisebox{-0.5ex}{$x$}}{\raisebox{0.5ex}{$\lambda \kern+0.3pt$}} \kern+0.5pt \Bigr)
	\end{equation*}
	where $N \kern+1.0pt$ particles remain over some distance $x \kern+0.5pt$ with $N_0$ as the initial amount. Furthermore,
	\begin{equation*}
		P(x) \kern+0.3pt = 1 -
		\kern+0.9pt \exp \Bigl( -\frac{\raisebox{-0.5ex}{$x$}}{\raisebox{0.5ex}{$\lambda \kern+0.3pt$}}\kern+0.5pt \Bigr)
	\end{equation*}
	gives the probability of a particle having been scattered after travelling $x \kern+0.5pt$ length units. Similar steps
	for time instead of distance lead to the well known equation
	\begin{equation*}
		N(t) \kern+0.4pt = N_0 \exp \kern-2.0pt
		\raisebox{0.25ex}{\(\Bigl( \raisebox{-0.25ex}{\(-\frac{\raisebox{-0.5ex}{$t$}}{\raisebox{1.0ex}{$\tau \kern+1.0pt$}}\)}
		\kern+0.5pt \Bigr)\)}
	\end{equation*}
	describing exponential decay. It commonly appears in the context of radioactive materials but also applies to hadrons and leptons
	or more generally any quantity which decreases at a rate proportional to itself. Analogous to the previous case, a particle
	decays with probability
	\begin{equation*}
		P(t) \kern+0.3pt = 1 - \kern+0.9pt \exp \kern-2.0pt
		\raisebox{0.25ex}{\(\Bigl( \raisebox{-0.25ex}{\(-\frac{\raisebox{-0.5ex}{$t$}}{\raisebox{1.0ex}{$\tau \kern+1.0pt$}}\)}
		\kern+0.5pt \Bigr)\)}
	\end{equation*}
	before a time $t$ has passed and for $\tau$ as the mean lifetime. Translating this from rest frame to laboratory coordinates defines
	the decay timescale $t_\text{dec} \kern-0.6pt = \tau \kern+0.4pt \varGamma$ with a Lorentz factor $\varGamma = E / m$ via
	projectile energy and invariant mass. This is equivalent to a characteristic decay length given by
	$\lambda_\text{dec} \kern-0.5pt = v \kern+1.5pt t_\text{dec}$ where the velocity $v = c$ can be set for highly relativistic particles.
	Additionally, mean free path and cooling distance $\lambda_\text{cool} = (\kappa\sigma n)^{-1}$ refer to exchangeable concepts. Particles
	lose energy in every collision, which is the same as reducing temperature from a thermodynamics perspective. Dividing by the speed of
	light $c$ translates this expression to $t_\text{cool} = (\kappa\sigma n \kern+0.5pt c)^{-1}$ as a cooling timescale. Analogously, the
	distance $\lambda_\text{dec} \kern-0.5pt = c \kern+0.8pt \tau E / m$ can be rewritten as
	$t_\text{dec} \kern-0.6pt = \tau E / m$ in units of time. Substituting into the decay formula yields a cooling factor
	\begin{equation}
		\mathscr{C} = 1 - \kern+0.9pt \exp \biggl( -\frac{t_\text{cool}}{\raisebox{0.5ex}{$t_\text{dec}$}} \biggr)
		\label{eqn:cooling}
	\end{equation}
	that rescales spectra from direct production to account for decay processes taking place after collisional energy losses have occurred.
	This is an essential mechanism for the hypothesis that neutrinos from charm dominate pion and kaon contributions at high energies.
	While longer lived particles experience significant cooling thanks to time dilation, charmed hadrons decay promptly in comparison,
	resulting in a neutrino flux that directly traces the underlying hadronic population. One requirement for the validity of \eqref{eqn:cooling}
	is that $\lambda_\text{dec} \kern-0.3pt \ll d \kern+0.5pt$ holds, with $d \kern+0.5pt$ measuring the target field size to ensure decays
	occur exclusively inside this region.
	\enlargethispage*{\baselineskip}\newpage
\end{spacing}

\begin{spacing}{0.985}
	Another quantity relevant to the context at hand is an optical depth $\mathscr{O} = d / \lambda$ that represents the distance
	$d \kern+0.5pt$ to mean free path $\lambda$ ratio. Inserting its definition yields a factor
	\begin{equation}
		\mathscr{O} = \kappa\sigma n \kern+0.75pt d \: ,
		\label{eqn:optical}
	\end{equation}
	which is used as a first order multiplicative modification to account for attenuation effects on the spectra. This can be understood
	by considering that higher target medium densities and larger scattering cross sections over longer distances lead to more interactions
	with the projectile. 
	
	
	
	\section{Multimessenger Astronomy}
	\label{sec:multimessenger}
	
	A recent extension of traditional astronomy, the field of multimessenger astrophysics continues to evolve whilst providing otherwise
	inaccessible insights via the combination of complementary information carried by multiple messenger species. This section begins by
	expanding on physical concepts alluded to in Chapter \ref{ch:introduction} before continuing to describe the available multimessenger
	candidates in some detail. For a more extensive overview, refer to \cite{Meszaros_2019}.
	
	
	
	\subsection{Magnetic Field Scales}
	\label{sub:fields}
	
	Particles carrying electrical charges $Q = Ze \kern+0.5pt$ in orthogonal motion at velocity $v \kern+1.0pt$ to a homogenous magnetic
	field $B$ are acted on by the Lorentz force $F = QvB \kern+0.5pt$ that produces a gyrating motion. With the elementary charge $e$ and
	$\varGamma$ as a Lorentz factor, this must balance a relativistic centripetal force $F = m \varGamma v^2 \kern-0.5pt / R$ with Larmor
	radius $R$ on the resulting circular path. Rearranging variables and identifying $p = m \varGamma v \kern+0.5pt$ leads to the solution
	\begin{equation*}
		R = \frac{p}{\raisebox{0.3ex}{$QB \kern+0.3pt$}}
	\end{equation*}
	on which the magnetic field extent $D$ imposes $R < D$ as a condition. In case of highly relativistic energies, one can substitute
	$E = \kern-0.8pt pc$ for the momentum to obtain an inequality $E < QcBD \kern+0.5pt$ from the above considerations. Realistic astrophysical
	magnetic fields are not ordered, instead following turbulences that travel through the plasma. To include the effects of moving scattering
	centers, a factor $\kern-0.5pt \beta \kern+0.5pt$ proportional to the Alfvén velocity is included, giving
	\begin{equation*}
		E < Ze \kern-0.3pt \beta \kern+0.9pt cBD
		\label{eqn:hillas}
	\end{equation*}
	as the Hillas criterion. It is named after its description in \cite{Hillas_1984} and connects source region sizes to the strength of
	prevailing magnetic fields, with observed cosmic ray energies as a constraint.
	
	
	
	\subsection{High Energy Cutoff}
	\label{sub:cutoff}
	
	At very high energies, protons and nuclei interact with cosmic photons, which can be blueshifted up to extreme gamma regimes due to
	the Doppler effect. In these processes, the photon spin is absorbed to produce a delta resonance $\Delta^+$ representing an excited
	proton state.
	\enlargethispage*{\baselineskip}\newpage
\end{spacing}

This decays almost immediately to pairs of nucleons and pions, leading to probable reaction channels
$p\gamma \rightarrow \kern-0.1pt n\smash{\pi^+}$ or $p\gamma \rightarrow \kern+0.2pt p \smash{\pi^0}$ and resulting in an altered
particle momentum. For the case of \emph{Cosmic Microwave Background} (CMB) radiation, a close to perfect black
body spectrum
\begin{equation*}
	\frac{\raisebox{-0.3ex}{$dn$}}{\raisebox{0.15ex}{$d\epsilon$}} (\epsilon) \kern+0.2pt =
	\frac{\raisebox{-0.3ex}{$\epsilon^2$}}{\pi^2 \hbar^3 c^3 \bigl( e^{\kern+1.0pt \epsilon / k_B T} - 1 \bigr)}
\end{equation*}
is given by \cite{Gaisser_2016} as the number density. From CMB temperatures of $\kern+0.5pt T \kern+0.5pt = \qty{2.725}{\kelvin}$ follows that
significant photon numbers exist with energies around $\epsilon \kern+0.5pt = \qty{e-12}{\giga\electronvolt}$ in the CMB rest frame. 
To find a threshold in center of mass coordinates, the scalar product is used for
\begin{equation*}
	s \kern+0.3pt = \bigl( p_{\kern-0.5pt p} + p_{\kern+0.5pt \gamma} \bigr)^2 =
	p_{\kern-0.5pt p}^2 + p_{\kern+0.5pt \gamma}^2 + 2 p_{\kern-0.5pt p} p_{\kern+0.5pt \gamma} =
	m_p^2 \kern+0.8pt c^4 \kern-0.2pt + 2 E \epsilon = \kern-0.2pt M^2c^4 \: ,
\end{equation*}
where $M \kern-0.3pt = m_p + \kern+0.3pt m_\pi$ and $m_\gamma = 0$ as well as head-{\kern+0.2pt}on collisions have been assumed. The solution
\begin{equation*}
	E = \Bigl( \kern-1.0pt \bigl( m_p + m_\pi \bigr)^2 c^4 \kern-0.2pt - m_p^2 \kern+0.8pt c^4 \Bigr) / (2\epsilon)
\end{equation*}
corresponds to roughly $E \kern+1.0pt = \kern-0.5pt 10^{1 \kern-0.3pt 1} \kern+1.5pt \unit{\giga\electronvolt}$ for the so-called
GZK cutoff. Energies that exceed this value when viewed from CMB coordinates lead to significant losses, making the universe
opaque to such protons. If pair production $\gamma \kern-0.5pt \rightarrow \smash{e^+ \kern-0.5pt e^-}$ from proton interactions is
considered, one finds
\begin{equation*}
	E = \Bigl( \kern-1.0pt \bigl( m_p + 2m_e \bigr)^2 c^4 \kern-0.2pt - m_p^2 \kern+0.8pt c^4 \Bigr) / (2\epsilon)
\end{equation*}
or approximately $E \kern+1.0pt = \kern-0.5pt \qty{e+9}{\giga\electronvolt}$ by setting $M \kern-0.3pt = m_p + \kern+0.3pt 2m_e$
instead. Similar constraints apply when considering alpha particles and heavier nuclei, while electrons are limited mainly by Compton
down-{\kern+0.1pt}scattering. Additionally, there exist infrared and radio backgrounds as potentially relevant radiation fields. From
the multiple competing mechanisms for energy losses and momentum isotropization, it is as a consequence extremely challenging
to reliably interpret cosmic ray signals and practically impossible to reconstruct any information about specific sources.



\subsection{Stochastic Acceleration}
\label{sub:acceleration}

From the introduction of theoretical limits in Sections \ref{sub:fields} and \ref{sub:cutoff} arises the question of how
particles reach these high energy regimes in the first place. Due to its relatively simple derivation and surprising accuracy
in the description of measured spectra, probabilistic collisions are often viewed as one of the most plausible mechanisms
responsible for cosmic ray acceleration. The general case is described in \cite{Longair_2011} and supposes that for each
collision, particles gain some energy proportional to a constant factor $\eta$ and remain in the accelerating region with
probability $\varsigma$ on average. With initial conditions $N_0$ for the number of particles and $E_0$ as the mean
energy, this results in $N \kern+0.9pt = N_0 \varsigma^k$ and $E \kern+0.9pt = E_0 \eta^k$ after $k$ collisions. Using
$\ln x^k = k\ln x$ in
\begin{equation*}
	\frac{\ln (N \kern+1.0pt / N_0)}{\ln (E / E_0)} = \frac{\ln(\varsigma)}{\ln(\eta)}
\end{equation*}
eliminates the exponent and allows terms to be rearranged to solve for the desired quantity.

\enlargethispage*{\baselineskip}\newpage

\begin{spacing}{0.990}
	From this follows a connection between energy and number of particles
	\begin{equation*}
		N \kern+0.9pt = N_0 \left( \frac{\raisebox{-0.3ex}{$E$}}{\raisebox{0.3ex}{$E_0$}} \right)^{\ln(\varsigma) / \kern-0.5pt \ln(\eta)}
	\end{equation*}
	that incidentally obeys a power law, which is an almost ubiquitous feature
	observed in cosmic ray physics. One can interpret the above expression as an integrated spectrum, leading to
	\begin{equation*}
		\frac{\raisebox{-0.3ex}{$dN$}}{\raisebox{0.3ex}{$dE$}} = \frac{\raisebox{-0.3ex}{$N_0$}}{\raisebox{0.3ex}{$E_0$}}
		\left( \frac{\raisebox{-0.3ex}{$E$}}{\raisebox{0.3ex}{$E_0$}} \right)^\alpha
	\end{equation*}
	for the differential case with a characteristic spectral index
	\begin{equation*}
		\alpha = \frac{\ln(\varsigma)}{\ln(\eta)} - 1
	\end{equation*}
	constrained by $\ln(\varsigma) / \kern-0.5pt \ln(\eta) < 0$ due to $\varsigma < 1$ and $\eta > 1$ as implied per the definitions.
	
	The basic case of \emph{Diffusive Shock Acceleration} (DSA) considers shock fronts that propagate with velocity
	$\kern-0.5pt \beta = v / c \kern+1.0pt$ in a fully ionized gas. Requiring momentum isotropization without significant energy losses
	on each side of the discontinuity results in $\ln(\varsigma) / \kern-0.5pt \ln(\eta) = -1$ and a corresponding spectral dependence
	$dN \kern+0.5pt / dE \propto \kern-0.5pt E^{\kern+0.5pt -2}$ that is discussed by \cite{Longair_2011} as well. A slightly steeper
	index around $\alpha = \num{-2.5}$ can be derived when nonlinear effects are accounted for.
	
	Energy gain increasing linearly with $\kern-0.5pt \beta \kern+1.0pt$ leads to this mechanism being categorized as first order Fermi
	acceleration, whereas the originally proposed formulation scales like $\beta^2$ or as second order. Though shocks exceed the local
	speed of sound in the astrophysical medium, relativistic velocities are typically not achieved. Consequently, ratios of $\beta \ll 1$
	mean that lower order processes are much more efficient in reaching high particle energies.


	\enlargethispage*{2\baselineskip}


	\subsection{Cosmic Rays}
	\label{sub:rays}
	
	Consisting of charged particles with strictly non-thermal origins, cosmic rays were first detected at high altitudes over a
	century ago. Since then, they have become a foundational piece in our understanding of astrophysical sources. Although their flux
	is dominated by protons, other nuclei as heavy as lead can contribute significant portions at high energies as well. Besides this
	baryonic component, there are also charged leptons adding to the total population. The exact ratios of these constituents contain
	information about source compositions, though any reconstruction of individual sources is likely impossible. This is a result
	of cosmic magnetic fields that scramble the directions of charged particles, leading to a largely isotropic distribution. At the
	same time, it is possible to constrain the size and field parameters of sources based on the energy of cosmic rays. For this
	purpose, Section \ref{sub:fields} provides the so-called Hillas condition. In combination with the characteristic broken power
	law spectrum, one concludes that most cosmic rays are of galactic origin, except for the highest energies, which require
	extragalactic sources \cite{Tjus_2020, Drury_2012}. Beyond these ranges, there appears to be a sharp drop that agrees well with
	Section \ref{sub:cutoff} in the case of a GZK cutoff. Both the spectral shape and the extreme energies are also consistent with
	a modified DSA scenario as described in Section \ref{sub:acceleration} for cosmic rays \cite{Becker_2008}.
	\newpage
\end{spacing}



\subsection{Neutrinos}
\label{sub:neutrinos}

Their weakly interacting nature introduced in Section \ref{sub:hadrons} is precisely what makes neutrinos so interesting. After
being produced, they penetrate most obstacles that would be completely opaque to any other messenger particle. This comes at the
cost of being notoriously difficult to detect, of course, but opens up the opportunity of measuring otherwise inaccessible processes
occurring deep inside astrophysical sources. Several different populations are either known or hypothesized, for example
those of solar origin, neutrinos from core-collapse supernovae, stellar remnants or active galaxies that are discussed in this thesis,
as well as neutrinos from pion decay implied in Section \ref{sub:cutoff} after GZK photodissociation. Additionally, a primordial
neutrino background is expected to exist with a history analogous to that of CMB formation, in which photons decoupled
from matter during the recombination epoch \cite{Becker_2008}. All of these astrophysical and cosmological prospects as well as the
previously mentioned flavor oscillations make neutrino astronomy a promising field of research \cite{Meszaros_2019}.



\subsection{Photons}
\label{sub:photons}

With visual observations being conducted for millennia, photonic signals undoubtedly form the most ancient type of astronomy.
Advances in optical instrumentation have driven revolutionary progress over the last centuries, culminating in modern detectors
unlocking wavelengths across the entire electromagnetic spectrum. Especially radio and gamma frequencies have the potential of
contributing to multimessenger astrophysics, with radio waves having the ability to resolve regions obscured by dust, and gamma
photons being closely associated with hadronic processes in a similar manner as neutrinos, for instance. Fundamental limitations
of astrophysical photon signals consist of a cutoff at very high energies from pair production with CMB photons similar to
Section \ref{sub:cutoff} for protons and nuclei, as well as a gamma ray horizon referring to the impedance caused by
interactions with extragalactic background light \cite{Dominguez_2013}. Detection is further constrained by the atmospheric
window, which is not transparent for all electromagnetic wavelengths.



\subsection{Gravitational Waves}
\label{sub:gravitational}

Setting aside the possibility of gravitons as hypothetical mediating particles, gravitation seems to be a fundamentally continuous
phenomenon, in the same way as classical theories like that of electromagnetism by Maxwell. According to Einstein, masses lead to
distortions of spacetime that propagate at the speed of light. From this perspective, it readily follows that especially massive
systems such as black hole or neutron star binaries emit gravitational waves through their movement. These excitations have an
advantage compared to other messengers in that they experience essentially no absorption or scattering by matter. Alongside unimpeded
insights into highly energetic merger events, there should also exist a gravitational background radiation containing information about
the very early universe.


\newpage\input{amend/special-head}


\section{Astrophysical Sources}
\label{sec:sources}

The picture of particle physics building up to multimessenger astronomy will now be realized in the context of plausible astrophysical
neutrino sources \cite{Becker_2008}. In the following section, scenarios for a magnetar as well as an AGN accretion disk are developed.



\subsection{Magnetars}
\label{sub:magnetars}

Neutron stars are stable remnants of massive stellar progenitors that are destroyed in violent core-collapse supernovae after having
depleted their fuel reserves. Instead of nuclear fusion, the remaining neutron star is supported against further gravitational
contraction by degeneracy pressures. These result from the Pauli exclusion principle that prevents fermions from occupying identical
quantum states. While white dwarfs rely primarily on electrons for this mechanism, neutron stars depend on the repulsion among
nucleons, placing them among the most dense macroscopic objects in the universe. Due to angular momentum of the progenitor
star being conserved in the remnant, the compression of stellar material immediately before a supernova explosion leads to
extremely fast rotational periods. In addition to that, conservation of magnetic energy implies large magnetizations as well.
The authors in \cite{Thompson_1993} discuss this so-called fossil field hypothesis, but argue that strong convective flows
during formation or inside the very young neutron star are more likely culprits. These would also eliminate any prior correlation
between the axis of rotation and the field orientation. Because neutron star interiors are assumed to be superconducting fluids,
dynamo effects can amplify and sustain any initial magnetic fields \cite{Haskell_2018}. If such a magnetized rotator emits misaligned
radio jets, their sweeping motion will be observed as a pulsating radio source, hence the name pulsar. In case of extreme magnetic
fields, the respective neutron star is further classified as a magnetar. The decay of these fields is generally modeled
as exponential or according to a power law, but can be approximated as being static for the shorter timescales relevant to the present
thesis \cite{Sang_1990}. Surrounding newborn neutron stars are clouds of ejected material, which serve as target fields. Modeling this
to be a spherical region of homogenously distributed matter at each time~$t \kern+0.5pt$~after formation of the remnant leads to
a number density corresponding to the ratio
\begin{equation}
	\cramped{n_\text{ej} = 3 M_\text{ej} / \bigl( 4\pi r_\text{ej}^3 m_p \bigr)} \kern+0.5pt
	\label{eqn:number-density}
\end{equation}
for all ejecta mass $M_\text{ej}$ consisting of protons with $m_p$ masses. By asserting a radial expansion at constant
$\beta_\text{ej}$ velocities, one finds the ejecta radius
\begin{equation}
	r_\text{ej} = \kern-0.4pt \beta_\text{ej} \kern+0.1pt c \kern+1.0pt t
	\label{eqn:ejecta-radius}
\end{equation}
as a function of time. Inside this plasma, free charge carriers weaken the otherwise strong fields, allowing synchrotron losses to be
neglected \cite{Carpio_2020}. In vacuo, these would significantly reduce the energy of injected protons, thereby preventing production
of highly energetic hadrons and neutrinos. Furthermore, typical densities listed in Section \ref{sub:production} are sufficiently low for
neutrino attenuation effects to be neglected, and a comparison of the optical depth \eqref{eqn:optical} for protons and photons shows that
thermal radiation fields are unlikely to influence the overall results \cite{Carpio_2020}.


\enlargethispage*{2\baselineskip}\newpage\input{amend/normal-head}


\subsection{Spindown}
\label{sub:spindown}

To explain observations of rapidly spinning neutron stars or pulsars, there has to exist some mechanism by which rotational energy is
lost. Reference \cite{Alvarez_2004} gives a brief overview of possible radiation candidates such as gravitational quadrupole or
higher order electromagnetic moments. Because this thesis is limited in its scope and concerns itself with the acceleration of
electric charges, a pure magnetic dipole approach will be adopted. For more compact and convenient notation, Gaussian units are used.

An idealized view like in \cite{Deutsch_1955} models stars as sharply bound and uniformly magnetized spheres, leading to
$\cramped{\mu \kern-0.1pt = \kern-0.6pt R_\text{ns}^3 B_\text{ns} \kern+0.1pt / 2}$ for the external magnetic moment \cite{Jackson_1999}.
Parameters $R_\text{ns}$ and $B_\text{ns}$ measure stellar radius and polar magnetic flux density, respectively. For a rotating
dipole in vacuo,
\begin{equation}
	L = \frac{2\mu^2 \omega^4 \kern-1.0pt \sin^2 \kern-2.0pt \chi \kern+1.0pt}{3c^3}
	\label{eqn:vacuum}
\end{equation}
is the exact expression for radiant power derived in \cite{Jackson_1999} as a standard reference, where a static angle
$\kern-1.0pt \chi \kern+1.0pt$ between the dipole field and rotational axis with angular frequency $\omega$ is assumed. If
a force{\kern+0.2pt}-free limit is applied instead, variational calculations \cite{Gruzinov_2006} and simulations
\cite{Spitkovsky_2006} indicate
\begin{equation}
	L = \frac{\mu^2 \omega^4 \raisebox{0.15ex}{\( \bigl(
	\raisebox{-0.15ex}{\( 1 \kern-0.7pt + \sin^2 \kern-2.0pt \chi \kern+1.0pt \)} \bigr) \)} }{c^3}
	\label{eqn:ffe-mhd}
\end{equation}
as an appropriate expression of the luminosity. This is more accurate than the previous result, as it has been
shown by \cite{Goldreich_1969} that the surroundings of a neutron star cannot support a vacuum but must instead be filled
with a plasma of charge carriers originating from instabilities at the surface. However, one quickly encounters the problem that
$\bm{E} \cdot \kern-1.0pt \bm{B} = \kern-0.5pt 0$ as a condition of force{\kern+0.2pt}-free magnetospheres prevents particle acceleration.
This can be overcome by introducing deviations from such global solutions on local scales as discussed in \cite{Li_2012} and \cite{Gralla_2019}.

Spindown follows an exponential energy decay $E = \kern-0.3pt E_0 \kern-0.2pt \exp \kern-2.0pt
\smash{\bigl( - \kern+0.5pt t \kern+0.1pt / \kern+0.2pt t_\text{sd} \kern+0.5pt \bigr)}$ with the time derivative
\begin{equation}
	\dot{E} = - E / t_\text{sd}
	\label{eqn:exp-derivative}
\end{equation}
and $t_\text{sd}$ as a characteristic timescale. Assuming exclusively rotational energy
\begin{equation}
	E = \kern-0.5pt I \kern+0.5pt \omega^2 \kern-1.5pt / 2 \: ,
	\label{eqn:quad}
\end{equation}
where $I \kern+0.5pt$ gives the neutron star moment of inertia, leads to
\begin{equation}
	\dot{E} = \kern-0.5pt I \kern+0.5pt \omega \kern+0.5pt \dot{\omega}
	\label{eqn:quad-derivative}
\end{equation}
as an equally valid expression of the energy loss rate. Equations \eqref{eqn:vacuum} and \eqref{eqn:ffe-mhd} can be generalized via
$L = \kern-0.5pt K \omega^4$ by using a coefficient $K$ containing all information except the \mbox{dipolar $\omega^4$ dependence.}

\newpage

Identifying this with \eqref{eqn:exp-derivative} through $\dot{E} = -L$ and evaluating at $t \kern-0.3pt= 0$ yields
$L_0 \kern-0.3pt = \kern-0.2pt K \omega_0^4$ as well as the decay rate $\dot{E}_0 \kern-0.3pt = -I \omega_0^2 / 2 \kern+0.4pt t_\text{sd}$
from inserting \eqref{eqn:quad} as the energy, which can be combined to find
\begin{equation*}
	t_\text{sd} = I \kern+0.5pt / 2K\omega_0^2
	\label{eqn:spindown-time}
\end{equation*}
for the spindown time. It further follows from equating \eqref{eqn:quad-derivative} to the luminosity at any $t \kern+0.5pt$ that
\begin{equation*}
	I \kern+0.5pt \dot{\omega} = -K \omega^3
\end{equation*}
must hold. This represents a special case of the power law differential equation $\dot{\omega} \propto \omega^n$ with a so-called
braking index $n = 3$ typical of a rotating magnetic dipole. As a comparison, one would find $n = 5$ in case of a quadrupole, indicating
that $n$ is characteristic of the specific energy loss mechanism involved \cite{Alvarez_2004}. After separating variables and integrating,
one obtains
\begin{equation*}
	\omega = \omega_0 \smash{\bigl( 1 + \kern+1.5pt t \kern+0.1pt / \kern+0.2pt t_\text{sd} \kern+0.5pt \bigr)^{-1/2}}
	\label{eqn:frequency}
\end{equation*}
for the frequency as a function of time and spindown scale.

Another important concept for the description of neutron star environments is the so-called light cylinder. It represents the boundary
beyond which magnetic field lines can no longer corotate with the central body without exceeding the speed of light, defining
$R_\text{lc} \kern-0.6pt = c / \kern-0.4pt \omega$ as its radius. Due to the resulting lag, magnetic fields
are open, providing a potential mechanism for the acceleration of particles to high energies \cite{Goldreich_1969}.

With an open field magnetic flux $\psi = \mu / R_\text{lc}$ follows $\phi = \psi / R_\text{lc}$ as the difference in potential energy.
From this, a charge $e$ reaches energies $E^M = \eta e \phi \kern+0.5pt$ with an efficiency $\eta < 1$ limiting the acceleration.
Expanding these terms and including the dependence on time,
\begin{equation}
	E^M(t) = \frac{\eta \kern+0.5pt e B R^3 \omega_0^2}{2c^2 \bigl( 1 + \kern+1.5pt t \kern+0.1pt / \kern+0.2pt t_\text{sd} \kern+0.5pt \bigr)}
	\label{eqn:mono}
\end{equation}
states this explicitly. The superscript $M$ denotes a monochromatic energy distribution.



\subsection{Active Galactic Nuclei}
\label{sub:nuclei}

As the bright centers observed in some galaxies, these compact cores are emitters of extreme energies, making them prime
candidates for UHECR acceleration. It is accepted that AGN are driven by SMBH central engines. These black holes are described
by the Kerr metric, giving them a spin and a corresponding angular momentum \cite{Visser_2008}. Through interactions of
infalling material and magnetic fields, the rotational energy can be extracted to form highly relativistic jets \cite{Blandford_2019}.
In the plane of rotation surrounding the central mass, one usually finds an accretion disk emitting luminous thermal radiation,
a thorough review of which is given by \cite{Abramowicz_2013}. The central SMBH is enveloped by a hot plasma of low density,
defining the so-called corona region.

\newpage

Included in \cite{Beckmann_2013} is a brief overview of the unified AGN model that assumes a generally similar structure for all active
galactic cores. In this view, distinct AGN types are differentiated
by their radio emissivity and total luminosity, as well as the relative orientation in the direction of an observer. The
last feature predicts the measured effects of broad and narrow line regions or the obstruction by a dusty torus that is situated
around the outer accretion disk. Potential sources of high-{\kern+0.3pt}energy neutrinos in the complex AGN environment are discussed
in \cite{Murase_2023}. This thesis assumes a general DSA mechanism, which injects energetic particles onto a static accretion disk as
the proton target. The shocks involved in this acceleration have several possible origins, such as the corona or the disk itself. Other
options are jet shocks that travel towards the inner region or starburst events in the surrounding galaxy as tested in \cite{Eichmann_2022}.
One more critical assumption is that of vanishing magnetic fields inside the AGN disk, which is necessary to neglect synchrotron losses.
At first order, this is justified by the same reasoning given for the magnetar setting.



\subsection{Accretion Disks}
\label{sub:accretion}

Diffusely distributed material in orbit around a central massive object naturally produces flat structures. This is a
consequence of gravitational forces being compensated in the radial plane by rotational effects while matter is relatively
free to collapse in the axial direction. Compression and friction during inward spiraling heat the disk, thereby emitting
intense thermal radiation. Settings where these phenomena likely occur are the protoplanetary disks that surround newly formed
stars or accretion flows on a SMBH at the center of a galactic core region, among others. The following describes some related effects
that are also discussed in \cite{Longair_2011}.

In Newtonian physics, the gravitational force at distance $R \kern+0.5pt$ from a mass $M \kern+0.5pt$ is given as
\begin{equation}
	F = \frac{\raisebox{-0.3ex}{$GMm$}}{R^2}
	\label{eqn:newton}
\end{equation}
for particles with masses $m$ orbiting the central object. The resulting acceleration is therefore of magnitude
$g \kern-0.3pt = GM / \smash{R^2}$ with vertical component $\ddot{z} = \kern-0.6pt g \sin(\vartheta)$ and
$\vartheta \kern+0.5pt$ as the angular elevation.~In case of $\kern+0.5pt \vartheta \ll \pi / 2$ one can approximate
$\sin(\vartheta) = \kern+0.3pt z / \kern-0.2pt R \kern+0.5pt$ to obtain
\begin{equation*}
	\ddot{z} = \frac{\raisebox{-0.3ex}{$GMz$}}{R^3}
\end{equation*}
which is related to pressure $p \kern+0.5pt$ and density $\kern-0.5pt \rho \kern+0.5pt$ via
$d \kern-0.1pt p / \kern-0.1pt dz = -\rho \ddot{z} \kern+0.5pt$ for a stable hysdrostatic equilibrium to exist. Using the speed
of sound $\smash{u^2} \kern-0.7pt = d \kern-0.1pt p / \kern-0.1pt d \kern-0.1pt \rho \kern+0.5pt$ allows reformulation to
\begin{equation}
	\frac{d \kern-0.1pt \rho}{\raisebox{0.2ex}{$dz \kern+1.0pt$}} =
	-\rho \kern+1.0pt \frac{\raisebox{-0.3ex}{$GMz$}}{u^2R^3}
	\label{eqn:density}
\end{equation}
and $\rho \kern+0.6pt = \kern-0.6pt \rho_0 \kern-0.5pt \exp \kern-1.0pt \smash{\bigl( -z^2 \kern-1.8pt / 2 h^2 \bigr)}$
after integration. According to this result, the height $h$ defined via
\begin{equation}
	h^2 = \frac{\raisebox{-0.3ex}{$u^2R^3$}}{\raisebox{0.2ex}{$GM \kern+1.5pt$}}
	\label{eqn:height}
\end{equation}
gives the standard deviation of the density, which follows a centered Gaussian distribution.

Assuming $M \gg m$ and circular Keplerian orbits, expression \eqref{eqn:newton} equals the
acting centripetal force, leading to $\smash{v^2} \kern-1.0pt = GM / R \kern+0.2pt$ for the orbital velocity. From writing
$\smash{h^2 \kern-0.5pt = u^2R^2 \kern-1.5pt / v^2}$ and the small angle approximation $h \ll R$ follows $u \ll v \kern+1.0pt$
as a necessary condition, meaning that the orbital velocity must greatly exceed the speed of sound specific to the medium. This
special case is called a thin disk and can further be extended to slim or even thick types of accretion structures.

Any luminous object radiating with spherical symmetry exerts a pressure
\begin{equation*}
	P = \frac{\raisebox{-0.3ex}{$L$}}{4\pi R^2 c}
\end{equation*}
as a function of luminosity $L \kern+0.5pt$ at distance $R \kern+0.5pt$ from the source. Suppose a gaseous cloud containing particles
with masses $m$ is falling towards the same bright central mass $M \kern+0.5pt$ due to gravitational attraction. During this process,
each particle experiences $F = \kern-0.3pt \varkappa \kern+0.7pt mP \kern+0.5pt$ as an opposing radiative force, where the opacity
$\kern-0.3pt \varkappa \kern+0.7pt$ represents cross section per unit mass. Balancing \eqref{eqn:newton} leads to
\begin{equation*}
	L = \frac{\raisebox{-0.3ex}{$4\pi GM c \kern+0.4pt$}}{\raisebox{0.8ex}{$\varkappa$}}
	\label{eqn:eddington}
\end{equation*}
as the Eddington luminosity limit, beyond which additional matter is immediately blown away from the central object. By assuming that
the infalling material consists exclusively of ionized hydrogen, one can approximate $\varkappa \kern-0.1pt = m / \kern-0.5pt \sigma \kern+0.3pt$
with the Thomson cross section $\sigma \kern+0.5pt$ from scattering of electrons and $m$ as the proton mass. Though originally applied in the
context of stellar structures, this approach can also be used to describe accretion disks. If a compact object increases its mass
with rate $\dot{M} \kern+0.5pt$ due to accreting matter, some of the corresponding gravitational potential may be converted to
radiation. In terms of rest energy, a luminosity of $L = \eta\dot{M}c^2$ is obtained where a factor $\eta$ denotes the efficiency of this
mechanism. With \eqref{eqn:eddington} one finds
\begin{equation*}
	\dot{M} \kern-0.2pt = \frac{\raisebox{-0.3ex}{$4\pi GM$}}{\raisebox{0.8ex}{$\eta c \varkappa$}}
\end{equation*}
as an analogous steady state limit for the accretion rate, enforced by continuous balancing of radiation pressure and gravitational forces
as an intrinsic feedback process.
