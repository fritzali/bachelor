\chapter{Background}
\label{ch:background}

\section{Particle Physics}
\label{sec:particle}

The interaction and classification of elementary particles is generally described by the \emph{Standard Model} (SM) of particle physics. This
view includes three of the four known fundamental forces, namely electromagnetism as well as the weak and strong interactions. Each of these
sectors is characterized mathematically in the context of \emph{Quantum Field Theory} (QFT), which combines aspects from classical field theory,
special relativity and quantum mechanics to deliver some of the most precise predictions available today. Further gauge theoretical considerations
as well as the symmetries encoded in the unitary $\text{U}(1) \times \text{SU}(2) \times \text{SU}(3)$ group give rise to a self consistent model
of physics on microscopic scales.

Despite its impressive predictive power, there are some important observations which this approach has so far been incapable of explaining.
Perhaps most importantly, gravitational interactions as the fourth fundamental force are not incorporated at all. Due to the historical successes
in quantizing classical theories, it is widely believed that some form of quantum gravity is the most likely candidate for explaining the emergence
of macroscopic phenomena as described by \emph{General Relativity} (GR), another extremely accurate model of the universe at large.
However, in the energy regimes where such a hypothetical graviton is expected to become relevant, the formulation of the SM breaks down.


\section{Multimessenger Astronomy}
\label{sec:multimessenger}

\section{Astrophysical Sources}
\label{sec:sources}

\subsection*{Magnetars}
\label{sub:magnetars}

\subsection*{Active Galactic Nuclei}
\label{sub:agn}

