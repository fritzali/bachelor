\chapter{Background}
\label{ch:background}

While traversing space, messenger particles are subjected to many different influences, both in terms of type and magnitude, before being
detected at earth. Understanding this propagation requires a thorough comprehension of physical processes at all scales of application,
from production in astrophysical sources, through interactions with the vast radiation and matter fields that fill the cosmos, to entering
the solar system and finally the terrestial atmosphere.

The following sections provide an incomplete overview of aspects relevant to the treatment of this complex topic, as well as references
for further study on the various related subjects.

\section{Particle Physics}
\label{sec:particle}

In the modern view, interactions and categories of elementary particles are most accurately described by a construct called the
\emph{Standard Model}~(SM) of particle physics. Underlying this formalism is the mathematical framework known as
\emph{Quantum Field Theory}~(QFT) that combines and generalizes properties of \emph{Quantum Mechanics}~(QM) and
\emph{Special Relativity}~(SR) to produce an extremely precise description of microscopic reality.

Moreover, the SM is a gauge theory with $\text{U}(1) \times \text{SU}(2) \times \text{SU}(3)$ as the corresponding symmetry group.
Each component appearing in this direct product represents its own unitary subgroup and is associated with a different physical field.
Excitations in these fields are referred to as quanta and manifest as observable particles. Similar to regular QM there exist both
bosonic and fermionic fields that obey analogous commutation and anticommutation relations.

Since this work is mostly concerned with the statistical behaviour of large quantities instead of individual particle probabilistics,
a more detailed explanation is omitted except for certain phenomena important to the justification of some later assumptions.

\subsection*{Fundamental Interactions}
\label{sub:interactions}

Electromagnetism is described by \emph{Quantum Electrodynamics} (QED) and mediated by photons $\gamma$ with Abelian $\text{U}(1)$ symmetry,
whereas the weak force is carried by $Z$ and $W^\pm$ bosons related to the $\text{SU}(2)$ group. In terms of \emph{Electroweak Theory} (EWT)
all these particles actually result from symmetry breaking of virtual isospin and hypercharge fields, unifying $\text{U}(1) \times \text{SU}(2)$
under a single coherent structure. The previously massless gauge bosons then gain their masses via the Higgs mechanism, where only the
$\gamma$ is excluded and remains without mass.

\newpage

Notably, the weak interaction exclusively couples to left handed fermions, violating parity.

Chirality, Helicity, Decay, Mixing, Quarks, Color Charge

Carriers of the strong force are named gluons and arise from the $\text{SU}(3)$ symmetry group. 

Weak, Electromagnetic, Electroweak, Higgs

Parity Violation, Mixing

Limits, Dark Matter, Dark Energy, Neutrino Oscillation, Quantum Gravity

Vector Bosons, Scalar Boson, Boson Spins, Fermion Spins

General Relativity

\subsection*{Leptons}
\label{sub:leptons}

Neutrinos, Decays

\subsection*{Hadrons}
\label{sub:hadrons}

Confinement, Asymptotic Freedom, Residual Nuclear Force, Decays

\section{Multimessenger Astronomy}
\label{sec:multimessenger}

\subsection*{Gravitational Waves}
\label{sub:gravitational}

\subsection*{Charged Cosmic Rays}
\label{sub:rays}

\subsection*{Photons}
\label{sub:photons}

\subsection*{Neutrinos}
\label{sub:neutrinos}

\section{Astrophysical Sources}
\label{sec:sources}

\subsection*{Magnetars}
\label{sub:magnetars}

\subsection*{Active Galactic Nuclei}
\label{sub:nuclei}
