\chapter{Background}
\label{ch:background}



\section{Particle Physics}
\label{sec:particle}

The interaction and classification of elementary particles is currently most accurately described by the \emph{Standard Model}~(SM)
of particle physics. From combining and generalizing the properties of \emph{Quantum Mechanics}~(QM) and \emph{Special Relativity}~(SR)
emerges \emph{Quantum Field Theory}~(QFT) as the mathematical framework formalizing this construct. Certain features intrinsic to the~SM
require formulating additional gauge symmetries. Excitations occuring in the associated fields then correspond to
various particles, of which the following paragraphs provide a brief overview. Since this work mainly deals with the statistical behavior
of large quantities instead of individual particle probabilistics, a more detailed explanation is omitted.

Fundamental to the SM is a unitary $\text{U}(1) \times \text{SU}(2) \times \text{SU}(3)$ symmetry group, the generators of which can be
understood as representing bosonic field quanta with the physical function of mediating interactions between fermionic spinor fields.
Commutation relations equivalent to regular QM have to be fulfilled. Therefore, bosons must commute and have integer spin, whereas
fermions must anticommute and have half integer spin.

Carriers of the strong force arise from $\text{SU}(3)$ and are referred to as gluons. They couple to and themselves possess color
charges, enabling color changing and self interaction. The only type of elementary fermions with non vanishing color charge are quarks,
which exist in flavors of up and down, charm and strange, top and bottom, as well as the corresponding antiparticles. Appropriately,
\emph{Quantum Chromodynamics}~(QCD) is the name given to the QFT governing any interaction involving color charge.

\newpage

Weak, Electromagnetic, Electroweak, Higgs

Parity Violation, Mixing

Limits, Dark Matter, Dark Energy, Neutrino Oscillation, Quantum Gravity

General Relativity

\subsection*{Hadrons}
\label{sub:hadrons}

Confinement, Asymptotic Freedom, Residual Nuclear Force, Decays

\subsection*{Leptons}
\label{sub:leptons}

Neutrinos, Decays

\section{Multimessenger Astronomy}
\label{sec:multimessenger}

\subsection*{Gravitational Waves}
\label{sub:gravitational}

\subsection*{Charged Cosmic Rays}
\label{sub:rays}

\subsection*{Photons}
\label{sub:photons}

\subsection*{Neutrinos}
\label{sub:neutrinos}

\section{Astrophysical Sources}
\label{sec:sources}

\subsection*{Magnetars}
\label{sub:magnetars}

\subsection*{Active Galactic Nuclei}
\label{sub:agn}

